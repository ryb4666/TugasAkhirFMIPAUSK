%-------------------------------------------------------------------------------
%                            	BAB V
%               		KESIMPULAN DAN SARAN
%-------------------------------------------------------------------------------

\chapter{KESIMPULAN DAN SARAN}

\section{Kesimpulan}
	Berdasarkan analisis dan pengujian yang telah dilakukan pada aplikasi. Di dapatkan kesimpulan pada penelitian ini sebagai berikut:

	\begin{enumerate}
		\item Aplikasi ini terdiri dari 2 \textit{platform}, yaitu aplikasi berbasis android yang akan digunakan oleh pangkalan gas LPG untuk melakukan pecatatan tabung yang terjual dan verifikasi penerimaan pasokan tabung dan berbasis \textit{web} yang akan digunakan oleh agen gas LPG untuk melakukan monitoring pada penyaluran gas.
		\item Pada tahap pengembangannya, aplikasi ini menggunakan salah satu metode Agile yaitu Scrum
		\item Pada pengujian menggunakan metode \textit{black box}, aplikasi ini telah lulus secara valid pada setiap fitur-fiturnya yang berarti ada kesesuaian antara desain sistem dengan implementasi sistem.  
		\item Pada pengujian menggunakan metode \textit{white box}, aplikasi ini telah lulus yang berarti semua proses logika pada aplikasi berjalan dengan semestinya.
		\item Hasil analisis pengujian aplikasi dengan metode usability menunjukkan hasil yang baik, yaitu aplikasi memiliki interpretasi skor “Layak”. Hal ini berarti aplikasi ini bernilai baik dan dapat digunakan.
		
	\end{enumerate}


\section{Saran}

	Penelitian ini masih banyak kekurangan sehingga perlu dikembangkan agar menjadi lebih baik. Berikut adalah saran untuk penelitian ini:
	\begin{enumerate}
		\item Tampilan dari aplikasi Android dapat diperbaiki lagi pada saat mengambil barang agar lebih \textit{user friendly}. 
		\item UX (\textit{User Experience}) dari aplikasi android dapat di \textit{improve} agar proses pembelian tabung lebih cepat
		\item Pada aplikasi web, dapat ditambah fitur untuk mengimpor data penerimaan pasokan tabung via CSV. 
	\end{enumerate}

	
% Baris ini digunakan untuk membantu dalam melakukan sitasi
% Karena diapit dengan comment, maka baris ini akan diabaikan
% oleh compiler LaTeX.
\begin{comment}
\bibliography{daftar-pustaka}
\end{comment}
