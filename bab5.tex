%-------------------------------------------------------------------------------
%                            	BAB V
%               		KESIMPULAN DAN SARAN
%-------------------------------------------------------------------------------

\chapter{KESIMPULAN DAN SARAN}

\section{Kesimpulan}
	Kesimpulan pada penelitian ini adalah sebagai berikut:

	\begin{enumerate}
		\item Pengembangan aplikasi ini dirancang berdasarkan masalah atau \textit{problem-based analysis} dengan menggunakan \textit{problem frames}. 
		\item Untuk mencapai \textit{problem frames}, tahap yang dilakukan terdiri dari membuat \textit{user stories}, \textit{requirement}, menentukan domain dan \textit{share phenomena}.
		\item Aplikasi yang dikembangkan terbagi menjadi 2 bagian yaitu aplikasi web untuk pemilik kos dan admin dan aplikasi Android untuk pencari kos. 
		\item Pemilik kos dapat mempromosikan kos miliknya melalui aplikasi web jika diverifikasi oleh admin.
		\item Pencari kos dapat mencari dan memesan kos dari aplikasi Android tanpa harus berkeliling Banda Aceh untuk mencari informasi kos.
		\item Aplikasi Android yang dikembangkan menggunakan fitur QR Code untuk melihat informasi kos tanpa harus bertanya langsung pada pemilik kos.
		\item Berdasarkan pengujian \textit{usability} dengan metode \textit{System Usability Scale} (SUS), aplikasi web untuk pemilik kos mendapatkan skor SUS 72,5 sedangkan aplikasi Android untuk pencari kos mendapatkan skor 80,6, dimana kedua nilai tersebut masuk pada kategori \textit{Acceptable} yaitu dapat diterima oleh pengguna akhir.
		
	\end{enumerate}


\section{Saran}

	Penelitian ini masih banyak kekurangan sehingga perlu dikembangkan agar menjadi lebih baik. Berikut adalah saran untu penelitian ini:
	\begin{enumerate}
		\item Aplikasi yang digunakan oleh pemilik kos dan pencari kos tersedia dalam web, Android dan iOS.
		\item Tampilan dari aplikasi dapat diperbaiki lagi menjadi lebih menarik dan \textit{user friendly}. Agar dapat menaikkan \textit{SUS Score} terutama pada aplikasi web.
		\item Pada aplikasi Android, dapat ditambah fitur \textit{bookmark}. Sehingga tidak perlu mencari kos yang dimaksud di halaman Lihat Semua Kos. 
	\end{enumerate}

	
% Baris ini digunakan untuk membantu dalam melakukan sitasi
% Karena diapit dengan comment, maka baris ini akan diabaikan
% oleh compiler LaTeX.
\begin{comment}
\bibliography{daftar-pustaka}
\end{comment}
