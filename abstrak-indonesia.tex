\begin{abstractind}
Gas 3 Kilogram (Kg) LPG  (Liquified Petroleum Gas) telah menjadi barang pokok utama di rumah tangga Indonesia, sejak pemerintah telah membuat perubahan dari minyak tanah ke gas LPG 3 Kg. Dikarenakan perubahan ini, pemerintahan dapat melakukan penghematan pada kas negara. Namun sekarang ketersedian tabung gas LPG 3 kg semakin berkurang menyebabkan pertamina selaku memberlakukan kebijakan kepada Pangkalan Gas untuk wajib mengisi \textit{Logbook} Distribusi Tabung Gas 3 Kg. Dengan diberlakukannya kebijakan baru ini, ada beberapa permasalahan yang muncul seperti penerapan Logbook ini masih juga mengalami kendala karena pelaporan gas LPG 3 Kg masih di lakukan secara manual sehingga terkadang banyak data penyaluran gas yang salah atau tidak sesuai dengan stok di pangkalan, dalam hal registrasi pembeli baru masih dilakukan dengan cara mengisi formulir yang sangat banyak sehingga proses registrasi memerlukan waktu yang lama. Solusi yang dapat dilakukan yaitu membuat sebuah aplikasi yang mampu melakukan pendataan pada penyaluran gas LPG subsidi 3 Kg ke masyarakat dan melakukan verifikasi pada data tersebut. Aplikasi ini terdiri dari dua \textit{platform} yaitu \textit{platform web} dibuat dengan bahasa pemrograman java dan \textit{platform} android yang dibuat menggunakan ionic framework. \textit{Web} berfungsi untuk proses penambahan pangkalan gas baru,  penambahan pasokan tabung untuk pangkalan gas, dan mencetak laporan penyaluran tabung gas LPG 3kg. Aplikasi \textit{mobile} berfungsi untuk proses mencatat pembelian tabung gas, dan pendaftaran bagi pelanggan baru. Aplikasi-aplikasi ini dibangun dengan menggunakan metode scrum dengan jumlah iterasi sebanyak 5 iterasi. Pengujian \textit{whitebox} dan \textit{blackbox} dilakukan pada kedua platform untuk menguji fitur-fiturnya dan semua fitur \textit{Succesfully} berhasil. Pengujian \textit{Usability} dilakukan pada 8 responden menggunakan metode \textit{System Usability Scale} (SUS). Skor yang didapatkan untuk platform web adalah 77\% dan 78\% untuk platform android pada rentang 61\%-80\% dengan nilai interpretasi skor “Layak” . Berdasarkan hasil tersebut, aplikasi ini dapat terintegrasi dengan baik dan sesuai dengan kebutuhan setiap pengguna.


\bigskip
\noindent
\textbf{Kata kunci :} Agen Gas LPG, Gas LPG 3 Kg, Android, Ionic framework, Scrum, \textit{Blackbox}, \textit{Whitebox}, \textit{Usability testing}.
\end{abstractind}