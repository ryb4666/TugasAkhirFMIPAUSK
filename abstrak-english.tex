\begin{abstracteng}
\textit{3 Kg LPG (Liquified Petroleum Gas) gas has become a major staple item in indonesian households, ever since the goverment has made changes  from using kerosene to 3 Kg LPG. Due to this change, the goverment has been able to increase the state’s savings. However, due to the decreasing supply of these 3 Kgs LPG Gas Cylinders; Pertamina, as the main distributor has applied a new policy. This policy is targeted towards the gas distribution bases to fill out a logbook that would record the daily distribution of 3 Kgs gas cylinders. With this new policy being carried out, there are still few problems that happened in the field. One of them is the reporting process that is still done manually, therefore sometimes a loat of the data on the 3 Kg LPG gas cylinders are wrong or it does not confirm with the current stock on the distribution base. Every customer eligible to by the subsidized 3 Kg LPG cylinder must register must register in order to purchase the 3 Kg LPG Cylinders. The process is done by filling out forms. This taken up a lot of time. A solution that can be offered to solve this problem is by building an application that can collect the required data of the targeted community that will purchase the subsidized cylinders. The application will also be able to verify the data stored. This research aims to design and build the proposed application mentioned earlier. The application consist of two platforms, a web platform and a mobile platform using android. The web platform was built using Java Programming Language and the mobile platform using ionic framework. The web is responsible for process the addition of new gas bases, increase the supply of 3kg LPG Cylinders for gas bases, and print reports on the distribution of 3kg LPG Cylinders. The mobile app is responsible for process the recording of gas cylinder purchases, and registration for new customers. These applications are built using the scrum method with 5 iterations. whitebox and blackbox testings were conducted on both platform to test their features. All features were successfully tested. Usability was cunducted with 8 respondents, using System Usability Scale (SUS) method. The score of the questionnaire for the web platform was 77\% and 78\% for the android platform in the range of 61\%-80\% with a score of “Eligible” score.  Based on these results, this application is well integrated and in accordance with the needs of each other.}

\bigskip
\noindent
\textbf{\emph{Keywords :}} \textit{LPG Gas Agent, 3 Kg LPG Gas, Android, Ionic Framework, Scrum , Blackbox, Whitebox, Usability testing}
\end{abstracteng}