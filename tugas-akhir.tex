%
% Template Laporan Tugas Akhir Jurusan Informatika Unsyiah 
%
% @author  Kurnia Saputra 
% @version 1.0
% @since 03.02.2016
%
% Template ini telah disesuaikan dengan aturan penulisan tugas akhir yang terdapat pada dokumen Panduan Tugas Akhir FMIPA Unsyiah tahun 2010.
%


% Template pembuatan naskah tugas akhir.
\documentclass{jifhasil}

\tolerance=1
\emergencystretch=\maxdimen
\hyphenpenalty=10000
\hbadness=10000

%\usepackage[a4paper,left=14cm,right=3cm,top=3cm,bottom=5cm]{geometry}

% Daftar pemenggalan suku kata dan istilah dalam LaTeX
\include{hype.indonesia}

% Untuk prefiks pada daftar gambar dan tabel
\usepackage[titles]{tocloft}
\renewcommand\cftfigpresnum{Gambar\  }
\renewcommand\cfttabpresnum{Tabel\   }

\newcommand{\listappendicesname}{DAFTAR LAMPIRAN}
\newlistof{appendices}{apc}{\listappendicesname}
\newcommand{\appendices}[1]{\addcontentsline{apc}{appendices}{#1}}
\newcommand{\newappendix}[1]{\section*{#1}\appendices{#1}}

% Untuk hyperlink dan table of content
\usepackage[hidelinks]{hyperref}
\renewcommand\UrlFont{\rmfamily\itshape} %it's me!
\newlength{\mylenf}
\settowidth{\mylenf}{\cftfigpresnum}
\setlength{\cftfignumwidth}{\dimexpr\mylenf+2em}
\setlength{\cfttabnumwidth}{\dimexpr\mylenf+2em}

% Agar ada tulisan BAB pada TOC
\renewcommand\cftchappresnum{BAB } 
  \cftsetindents{chapter}{0em}{4.5em} %indenting bab
  \cftsetindents{section}{4.5em}{2em}
  \cftsetindents{subsection}{6.5em}{3em}
 
% Agar di TOC setiap angka bab/subbab diakhiri titik

\renewcommand{\cftsecaftersnum}{.}
\renewcommand{\cftsubsecaftersnum}{.}

% Agar setiap angka bab/subbab diakhiri titik
\usepackage{titlesec}
\titlelabel{\thetitle.\quad}

% Agar disetiap caption table dan gambar diakhiri titik
\usepackage[labelsep=period]{caption}

% Untuk Bold Face pada Keterangan Gambar
\usepackage[labelfont=bf]{caption}

% Untuk caption dan subcaption
\usepackage{caption}
\usepackage{subcaption}


% Agar bisa menggunakan warna LaTeX
\usepackage{color} %it's me!

% Agar table yang panjang bisa cut ke next page    %byRennyAdr
\usepackage{longtable}

% Untuk page landscape        %byRennyAdr
\usepackage{pdflscape}
\usepackage{lscape}

% Agar bisa bikin code snippet
\usepackage{listings, lstautogobble} %it's me!


% Warna pada code snippet Java
\definecolor{javared}{rgb}{0.6,0,0} % untuk strings
\definecolor{javagreen}{rgb}{0.25,0.5,0.35} % untuk comments
\definecolor{javapurple}{rgb}{0.5,0,0.35} % untuk keywords
\definecolor{javadocblue}{rgb}{0.25,0.35,0.75} % untuk javadoc

% Warna pada code snippet C/C++
\definecolor{mygray}{rgb}{0.4,0.4,0.4}
\definecolor{mygreen}{rgb}{0,0.8,0.6}
\definecolor{myorange}{rgb}{1.0,0.4,0}

% Sampul Depan
%-----------------------------------------------------------------
% Sampul Depan
%-----------------------------------------------------------------
\judul{RANCANG BANGUN APLIKASI PELAPORAN PENYALURAN GAS LPG 3KG PADA PANGKALAN GAS LPG BERBASIS ANDROID}

\judulinggris{\textit{DESIGN AND DEVELOPMENT OF ANDROID BASED 3KG LPG GAS DISTRIBUTION APPLICATION FOR LPG GAS BASE STATION}}

% nama lengkap
\fullname{Andri Darnius}

% NPM (Nomor Pokok Mahasiswa)
\idnum{1608107010057}

\degree{Sarjana Komputer}

\yearsubmit{Juli, 2019}

\program{Informatika}

\dept{Informatika}

% Pembimbing Pertama
\firstsupervisor{Rahmad Dawood, S.Kom., M.Sc.}
\firstnip{197203181995121001}

% Pembimbing Kedua
\secondsupervisor{Kurnia Saputra, S.T., M.Sc.}
\secondnip{198003262014041001}

% Ketua Jurusan
\kajur{Dr. Muhammad Subianto, S.Si., M.Si.}
\kajurnip{196812111994031005}

% Dekan Fakultas
\dekan{Dr. Teuku M. Iqbalsyah, S.Si., M.Sc.}
\dekannip{197110101997031003}

% tangal lulus proposal, seminar hasil atau sidang
\approvaldate{Rabu, 31 Juli 2019}

%-----------------------------------------------------------------
% End of Sampul Depan
%-----------------------------------------------------------------


% Awal dokumen
\usepackage{fancyhdr}
\usepackage{rotating}
% Untuk prefiks pada Daftar Program   
% byRennyAdr
\makeatletter
\begingroup\let\newcounter\@gobble\let\setcounter\@gobbletwo
\globaldefs\@ne \let\c@loldepth\@ne
\newlistof{listings}{lol}{\lstlistlistingname}
\endgroup
\let\l@lstlisting\l@listings
\AtBeginDocument{\addtocontents{lol}{\protect\addvspace{10\p@}}}
\makeatother
\renewcommand{\lstlistoflistings}{\listoflistings}
\renewcommand\cftlistingspresnum{Program~}
\cftsetindents{listings}{1.5em}{7em}

%tab didaftar pustaka -Indah
\setlength{\bibhang}{30pt}

\usepackage{pdfpages}

\begin{document}

\cover

\fancyhf{} 

\fancyfoot[C]{\thepage}


\approvalpage

%-----------------------------------------------------------------
% Disini kata pengantar
%-----------------------------------------------------------------
\preface % Note: \preface JANGAN DIHAPUS!

\vspace{0.5cm}
\begin{onehalfspace}
	
Puji dan syukur atas kehadirat Allah SWT yang telah melimpahkan rahmat dan karunia-Nya sehingga penulis dapat menyelesaikan Proposal Tugas Akhir dengan baik. Shalawat serta salam penulis hanturkan kepada Nabi Muhammad SAW yang telah membawa manusia dari zaman jahiliyah ke zaman yang penuh ilmu pengetahuan seperti saat ini.

Dalam pembuatan Proposal Tugas Akhir yang berjudul \textbf{Analisa Kebutuhan Aplikasi Pencarian Indekos Berbasis Android Menggunakan Problem-based Analysis}, penulis mendapatkan bimbingan dan bantuan dari berbagai pihak yang sangat berjasa, oleh karena itu pada kesempatan ini penulis mengucapkan terima kasih kepada:

\begin{enumerate}
\item Ayahanda dan Ibunda tercinta, yang tidak pernah lupa memberikan dukungan secara moril maupun materil, serta kakak dan abang yang mendukung dalam penulisan proposal ini.
\item Bapak Dr. Muhammad Subianto, M.Si. (Ketua Jurusan Informatika Fakultas Matematika dan Ilmu Pengetahuan Alam Universitas Syiah Kuala).
\item Bapak Nazaruddin, M.Eng.Sc., selaku dosen pembimbing pertama yang telah memberikan bimbingan dan arahan dalam penelitian ini.
\item Bapak Kurnia Saputra, M.Sc., selaku dosen pembimbing kedua yang telah memberikan bimbingan dan arahan dalam penelitian ini.
\item Bapak Kurnia Saputra, M.Sc., selaku koordinator TA Jurusan Informatika.
\item Miss Dalila Yunardi, B.Sc., M.Sc.,  selaku dosen pembimbing akademis.
\item Seluruh Dosen di Jurusan Informatika yang tidak bisa disebutkan satu-satu, atas ilmunya selama penulis kuliah di Informatika.
\item Renny, Dedek, Iren, Ira, Naya, Rehan, Aida, Kiki dan Adel yang telah menjadi tempat berbagi segala macam masalah dan solusi.
\item Keluarga IPA1 yang juga memberikan semangat untuk menyelesaikan proposal ini.
\item Seluruh teman-teman khususnya teman-teman jurusan Informatika angkatan 2014 yang telah memberi semangat serta saran dan kritiknya kepada penulis dalam menyelesaikan proposal penelitian.
\end{enumerate}

Dalam penulisan proposal, penulis menyadari bahwa penulisan proposal ini masih jauh dari kata sempurna, untuk itu mohon kritik dan saran yang bersifat membangun untuk laporan yang lebih baik lagi di masa yang akan datang.

\vspace{0.5cm}

\begin{tabular}{p{7.5cm}c}
&Banda Aceh, September 2018\\
&\\
&\\
&\textbf{Penulis}
\end{tabular}

\end{onehalfspace}


%-----------------------------------------------------------------
% TOC menggunakan single space
%-----------------------------------------------------------------
\begin{singlespace}
	\tableofcontents
\end{singlespace}

\addcontentsline{toc}{chapter}{DAFTAR ISI}
\listoftables
\addcontentsline{toc}{chapter}{DAFTAR TABEL}
\listoffigures
\addcontentsline{toc}{chapter}{DAFTAR GAMBAR}

%\renewcommand{\lstlistlistingname}{DAFTAR PROGRAM}
%\lstlistoflistings
%\addcontentsline{toc}{chapter}{DAFTAR PROGRAM}

\listofappendices
\addcontentsline{toc}{chapter}{DAFTAR LAMPIRAN}

%-----------------------------------------------------------------
% Daftar Singkatan 
%-----------------------------------------------------------------
%\include{daftar-singkatan}

%-----------------------------------------------------------------
% Disini awal abstrak
% Note: tidak ada abstrak untuk proposal
% Jika anda memilih style jifproposal, berikan comment untuk include abstrak di bawah ini.
%-----------------------------------------------------------------
\begin{abstractind}
Gas LPG 3 Kg sekarang telah menjadi kebutuhan utama bagi masyarakat di Indonesia sejak pemerintah melakukan konversi dari minyak tanah ke LPG (\textit{Liquified Petroleum Gas}) 3 kilogram (Kg) dan telah mampu memberikan penghematan yang signifikan pada kas negara. Namun sekarang ketersediaan tabung gas LPG subsidi 3 Kg semakin berkurang menyebabkan pertamina selaku distributor memberlakukan kebijakan kepada Pangkalan Gas untuk wajib mengisi \textit{Logbook} Distribusi Tabung Gas 3 Kg. Dengan diberlakukannya kebijakan baru ini, ada beberapa permasalahan yang muncul seperti penerapan \textit{Logbook} ini masih juga mengalami kendala karena pelaporan gas LPG 3 Kg masih di lakukan secara manual sehingga terkadang banyak data penyaluran gas yang salah atau tidak sesuai dengan stok di pangkalan, dalam hal registrasi pembeli baru masih dilakukan dengan cara mengisi formulir yang sangat banyak sehingga proses registrasi memerlukan waktu yang lama. Solusi yang dapat dilakukan yaitu membuat sebuah aplikasi yang mampu melakukan pendataan pada penyaluran gas LPG subsidi 3 Kg ke masyarakat dan melakukan verifikasi pada data tersebut. Aplikasi ini terdiri dari dua platform yaitu platform web dibuat dengan java dan android yang dibuat menggunakan \textit{ionic framework}. Proses pembuatan aplikasi ini menggunakan metode scrum. Pengujian aplikasi menggunakan 3 metode yaitu \textit{whitebox, blackbox} dan \textit{usability testing}. Hasil pengujian \textit{blackbox} bernilai valid untuk semua fitur yang diuji. Hasil pengujian \textit{whitebox} berhasil memberikan \textit{output} yang sesuai. Pengujian \textit{usability testing} dengan 8 orang responden mendapatkan hasil 77\% untuk web dan 78\% untuk android yang masuk dalam rentang 61\%-80\% dengan nilai interpretasi skor "Layak". Berdasarkan hasil tersebut, aplikasi ini terintegrasi dengan baik dan sesuai dengan kebutuhan setiap pengguna. 


\bigskip
\noindent
\textbf{Kata kunci :} Agen Gas LPG, Gas LPG 3 Kg, Android, Ionic framework, Scrum, \textit{Blackbox}, \textit{Whitebox}, \textit{Usability testing}.
\end{abstractind} %berikan comment jika proposal

\begin{abstracteng}
\textit{3 Kg LPG (Liquified Petroleum Gas) gas has become a major staple item in indonesian households, ever since the goverment has made changes  from using kerosene to 3 Kg LPG. Due to this change, the goverment has been able to increase the state’s savings. However, due to the decreasing supply of these 3 Kgs LPG Gas Cylinders; Pertamina, as the main distributor has applied a new policy. This policy is targeted towards the gas distribution bases to fill out a logbook that would record the daily distribution of 3 Kgs gas cylinders. With this new policy being carried out, there are still few problems that happened in the field. One of them is the reporting process that is still done manually, therefore sometimes a loat of the data on the 3 Kg LPG gas cylinders are wrong or it does not confirm with the current stock on the distribution base. Every customer eligible to by the subsidized 3 Kg LPG cylinder must register must register in order to purchase the 3 Kg LPG Cylinders. The process is done by filling out forms. This taken up a lot of time. A solution that can be offered to solve this problem is by building an application that can collect the required data of the targeted community that will purchase the subsidized cylinders. The application will also be able to verify the data stored. This research aims to design and build the proposed application mentioned earlier. The application consist of two platforms, a web platform and a mobile platform using android. The web platform was built using Java Programming Language and the mobile platform using ionic framework. The web is responsible for process the addition of new gas bases, increase the supply of 3kg LPG Cylinders for gas bases, and print reports on the distribution of 3kg LPG Cylinders. The mobile app is responsible for process the recording of gas cylinder purchases, and registration for new customers. These applications are built using the scrum method with 5 iterations. whitebox and blackbox testings were conducted on both platform to test their features. All features were successfully tested. Usability was cunducted with 8 respondents, using System Usability Scale (SUS) method. The score of the questionnaire for the web platform was 77\% and 78\% for the android platform in the range of 61\%-80\% with a score of “Eligible” score.  Based on these results, this application is well integrated and in accordance with the needs of each other.}

\bigskip
\noindent
\textbf{\emph{Keywords :}} \textit{LPG Gas Agent, 3 Kg LPG Gas, Android, Ionic Framework, Scrum , Blackbox, Whitebox, Usability testing}
\end{abstracteng} %berikan comment jika proposal
%-----------------------------------------------------------------
% Disini akhir masukan abstrak
%-----------------------------------------------------------------


% Caption untuk code snippet. it's me!
\renewcommand{\thelstlisting}{\arabic{chapter}.\arabic{lstlisting}}
\renewcommand*\lstlistingname{Program}

%-----------------------------------------------------------------
% Disini awal masukan untuk Bab
%-----------------------------------------------------------------
\begin{onehalfspace}

\fancyhf{} 
\fancyfoot[R]{\thepage}
\pagenumbering{arabic}


%-------------------------------------------------------------------------------
% 								BAB I
% 							LATAR BELAKANG
%-------------------------------------------------------------------------------

\chapter{PENDAHULUAN}

\section{Latar Belakang}
Gas LPG 3 Kg sekarang telah menjadi kebutuhan utama bagi masyarakat di Indonesia sejak pemerintah melakukan konversi dari minyak tanah ke LPG (\textit{Liquified Petroleum Gas}) 3 kilogram (Kg) yang dilakukan sejak 2007 dan telah mampu memberikan penghematan yang signifikan pada kas negara. Namun sekarang ketersediaan tabung gas LPG subsidi 3 Kg semakin berkurang menyebabkan pertamina selaku distributor memberlakukan kebijakan kepada Pangkalan Gas untuk wajib mengisi \textit{Logbook} Distribusi Tabung Gas 3 Kg. Pada \textit{Logbook} memiliki seluruh informasi tercatat lengkap mulai dari nama pembeli, alamat, jumlah pembelian, jenis pembelian, dan data agen yang menjadi distributor. Sehingga  pembeli yang ingin membeli gas subsidi ini harus menunjukkan kartu keluarga (KK) dan Kartu Tanda Penduduk (KTP) sehingga jumlah tabung yang di beli dapat dibatasi berdasarkan KK.
\par Dengan diberlakukannya kebijakan baru dalam hal penyaluran gas ini, ada beberapa permasalahan yang muncul seperti penerapan \textit{Logbook} ini masih juga mengalami kendala karena pelaporan gas LPG 3 Kg masih di lakukan secara manual sehingga terkadang banyak data penyaluran gas yang salah atau tidak sesuai dengan stok di pangkalan, dalam hal registrasi pembeli baru masih dilakukan dengan cara mengisi formulir yang sangat banyak sehingga proses registrasi memerlukan waktu yang lama. Memang dengan kebijakan yang baru dapat meminimalisir potensi kecurangan dalam membeli tabung gas LPG 3 Kg karena pangkalan akan mengecek nomor KK (Kartu Keluarga) dari si pembeli, namun proses pengecekan tersebut terkadang memakan banyak waktu dan ada potensi terjadinya kesilapan pada saat pengecekan.  Hal ini tentu saja akan menghambat pekerjaan dan membutuhkan waktu yang relatif lama sehingga seharusnya ada suatu pembaharuan sistem yang dilakukan sebagai langkah antisipasi.
\par Berdasarkan permasalahan tersebut, dibutuhkan sebuah aplikasi yang mampu mendata penyaluran gas LPG 3 Kg ke masyarakat. Sistem ini menggunakan platform Android untuk menjalankan aktivitas penyaluran tabung gas LPG seperti pembelian tabung, penerimaan tabung dari distributor, dan pelaporan gas 3 Kg. User dalam sistem ini adalah operator pangkalan gas LPG dan agen. Sedangkan, pembeli akan menggunakan KTP (Kartu Tanda Penduduk) sebagai alat identifikasi saat melakukan pembelian tabung dan operator akan melakukan verifikasi terhadap KTP dan melakukan pencatatan penjualan. Operator juga dapat melakukan register pembeli baru via aplikasi Android. Kebutuhan utama dari pengguna aplikasi ini adalah informasi penyaluran gas LPG 3 Kg dapat diakses secara cepat, mudah dan akurat dari perangkat mobile dengan layanan internet di dalamnya. Untuk mendukung kebutuhan akses informasi tersebut, dibutuhkan suatu aplikasi yang inovatif. Sistem operasi Android menawarkan kemampuan untuk membangun aplikasi yang inovatif serta bersifat open source.
\par Saat ini, pengembangan aplikasi Android telah banyak mengalami kemajuan yang pesat. Hadirnya Ionic yaitu \textit{framework} yang dikhususkan untuk membangun aplikasi mobile hybrid dengan HTML5, CSS dan AngularJS. Ionic menggunakan Node.js, SASS, dan AngularJS sebagai engine-nya. Ionic dilengkapi dengan komponen-komponen CSS seperti \textit{button, list, form, grids, tabs,} dan masih banyak lagi. Sehingga Ionic merupakan sebuah teknologi web yang bisa digunakan untuk membuat suatu aplikasi \textit{mobile}. Karena hybrid maka aplikasi hanya dibuat sekali tetapi sudah bisa dirilis di lebih dari satu \textit{platform} alias \textit{cross-platform}.
\par Kebanyakan orang menginginkan suatu kemudahan dalam penggunaan aplikasi. Kemudahan dalam menggunakan aplikasi dapat diukur dengan adanya pengujian tingkat \textit{usability}. Pengujian usability bertujuan untuk mengevaluasi produk aplikasi dengan secara langsung mengujinya kepada pengguna. Secara umum \textit{usability} mengacu pada metode bagaimana meningkatkan kemudahan dalam menggunakan suatu produk selama proses desain \citep{nielsen2012}. Hasil dari pengukuran tingkat usability nantinya bisa digunakan untuk mengembangkan aplikasi agar lebih baik dari sebelumnya.
\par Aplikasi ini diharapkan memiliki nilai usability yang baik, agar pengguna bisa dengan mudah menggunakan aplikasi dan dapat melayani kebutuhan pengguna dengan baik. Oleh karena itu fokus dari penilitian ini adalah untuk memberikan informasi penyaluran tabung gas LPG 3 Kg kepada pangkalan gas melalui smartphone berbasis Android. Diharapkan aplikasi ini dapat membantu pihak pangkalan dan agen gas LPG untuk menyalurkan gas LPG subsidi 3 Kg.

\newpage
\section{Rumusan Masalah}
Berdasarkan latar belakang yang telah diuraikan di atas, ada beberapa permasalahan yang akan dirumuskan pada penelitian ini, diantaranya:
\begin{enumerate}
	\item Apa saja kebutuhan yang diperlukan pengguna untuk merancang aplikasi pelaporan penyaluran Gas LPG 3 Kg.
	\item Bagaimana merancang aplikasi pelaporan penyaluran Gas LPG 3 Kg yang berbasis android.
	\item Bagaimana merancang aplikasi pelaporan penyaluran Gas LPG 3 Kg yang dapat dipakai dengan mudah dan cepat.
	\item Bagaimana menguji kelayakan aplikasi yang telah dibangun.
	.
\end{enumerate}

\section{Tujuan Penelitian}
Tujuan dari penelitian ini adalah sebagai berikut:
\begin{enumerate}
	\item Melakukan wawancara atau survei kebutuhan pengguna untuk mendapatkan fitur-fitur dari sistem yang akan dirancang dan dibangun.
	\item merancang aplikasi pelaporan penyaluran Gas LPG 3 Kg berbasis android.
	\item merancang aplikasi pelaporan penyaluran Gas LPG 3 Kg yang dapat dipakai dengan mudah dan cepat.
	\item menguji kelayakan dari aplikasi yang telah dibangun ntuk memastikan aplikasi telah sesuai dengan kebutuhan pengguna.
\end{enumerate}


\section{Manfaat Penelitian}
Manfaat dari penelitian ini adalah sebagai berikut:
\begin{enumerate}
	\item Meningkatkan efektivitas dalam semua proses terkait penyaluran tabung gas LPG 3 Kg.
	\item Menjadi bahan pembelajaran untuk mengembangkan sebuah \textit{mobile application} yang lebih baik dan berguna bagi masyarakat.
	\item Diharapkan aplikasi ini dapat mengawasi penyaluran tabung gas 3 Kg dan mempermudah masyarakat yang berhak untuk mendapatkan gas 3 Kg.
\end{enumerate}


% Baris ini digunakan untuk membantu dalam melakukan sitasi
% Karena diapit dengan comment, maka baris ini akan diabaikan
% oleh compiler LaTeX.
\begin{comment}
\bibliography{daftar-pustaka}
\end{comment}


%-------------------------------------------------------------------------------
%                            BAB II
%               TINJAUAN PUSTAKA DAN DASAR TEORI
%-------------------------------------------------------------------------------

\chapter{TINJAUAN KEPUSTAKAAN}                

\section{Struktur Penyaluran Gas LPG 3 Kg}
\par Dalam pendistribusian gas elpiji ke masyarakat, sepenuhnya dilakukan oleh Pertamina dengan sistem \textit{close loop  supply chain}, yaitu suatu aliran produk mulai dari konsumen, kembali ke pabrik untuk diproses ulang kemudian kembali lagi ke konsumen sebagai barang baru.
\par Dalam alur distribusi LPG 3 Kg, yang pertama adalah berasal dari Depot LPG. Kemudian dari Depot LPG, jalur berikutnya disebut SPPBE (Stasiun Pengisian dan Pengangkutan Bulk LPG ) yang dikelola oleh Pertamina dan pihak swasta, kemudian setelah itu paket LPG diterima oleh agen LPG dan selanjutnya sebagai ujung tombaknya disebut sub agen atau pangkalan LPG. Sub agen LPG inilah yang berhubungan langsung dengan pengecer, warung atau juga konsumen. seperti pada \ref{penyaluran}.
\begin{figure}[H]
	\centering
	\includegraphics [width = 10cm, height= 10cm]{gambar/struktur-penyaluran}
	\caption{Struktur Penyaluran Gas LPG}
	\label{penyaluran}
\end{figure}


\section{Gas Subsidi LPG 3Kg}
\par Sejak Pemerintah melakukan konversi dari minyak tanah ke gas LPG yang dilakukan pada tahun 2007 menyebabkan kebutuhan untuk tabung gas LPG semakin meningkat dan gas subsidi LPG 3 kg yang ditawarkan pemerintah di awal periode konversi, semakin berkurang. Ini dikarenakan jenis tabung ini dianggap lebih ekonomis di bandingkan dengan tabung gas dan banyak rumah tangga dan industri rumahan kecil yang masih memakai tabung gas jenis ini sebagai alat untuk memasak.
\begin{figure}[H]
	\centering
	\includegraphics [width = 6cm, height= 8cm]{gambar/tabung-gas}
	\caption{Tabung Gas LPG}
	\label{tabung}
\end{figure}

\par Maka dari itu pertamina selaku distributor telah memberlakukan kebijakan kepada konsumen yang ingin membeli gas subsidi ini harus menunjukkan kartu keluarga (KK) dan Kartu Tanda Penduduk (KTP) sehingga jumlah tabung yang di beli dapat dibatasi berdasarkan KK.

\section{Android}
\par Android merupakan sistem operasi yang dibangun untuk perangkat mobile. Komponen-komponen dari sistem operasi Android ditulis dengan bahasa pemrograman C atau C++, akan tetapi aplikasi pengguna yang digunakan untuk Android ditulis dalam bahasa pemrograman Java \citep{ableson2012android}. Android juga dapat diartikan sebagai sistem operasi perangkat seluler berbasis Linux yang menyediakan run time environment yang disebut dengan \textit{Android Runtime} (ART) yang telah dioptimasi untuk perangkat dengan sistem memori yang kecil. Karena Android merupakan platform open source, maka setiap orang bebas untuk membuat dan mengembangkan suatu aplikasi \citep{supardi2011}.
%\begin{figure}[H]
%	\centering
%	\includegraphics [width = 7cm, height= 6cm]{gambar/android}
%	\caption{Logo Android}
%	\label{android}
%\end{figure}



%-----------------------------------------------------------------------------%
\newpage
\section{Java}
Java pertama kali diluncurkan pada tahun 1995 sebagai bahasa pemrograman umum (\textit{general purpose programming language}) dengan kelebihan dia bisa dijalankan di web browser sebagai applet. Sejak awal, para pembuat Java telah menanamkan visi mereka ke dalam Java untuk membuat piranti-piranti yang ada di rumah (\textit{small embedded customer device}) seperti TV, telepon, radio, dan sebagainya agar dapat berkomunikasi satu sama lain. Karakteristik dari bahasa Java adalah sebagai berikut : \citep{Utama}
\begin{itemize}
	\itemsep0em
	\item Memiliki struktur Syntax yang sederhana
	\item Sangat berorientasi objek (OOP) dengan implementasi yang sangat baik sehingga
	kita bukan hanya belajar bagaimana membuat program yang baik (reusable,
	scalable, dan maintanable) tetapi juga kita belajar bagaimana cara berfikir yang
	baik untuk mengenali struktur masalah yang sedang kita hadapi.
	\item \textit{OpenPlatform}, \textit{Write Once Run Anywhere} (WORA), portabel atau \textit{multi platform}, program yang kita buat dapat dijalankan di Windows, Linux/Unix, Solaris, dan MacIntosh tanpa perlu diubah maupun di kompilasi ulang.
	\item Arsitekturnya yang kokoh dan pemrograman yang aman didukung oleh komunitas
	\textit{Open Source}
	\item Bukan sekedar bahasa tapi juga platform sekaligus arsitektur. Java mempunyai portabilitas yang sangat tinggi.
\end{itemize}

\section{Servlet}
Servlet merupakan salah satu bentuk aplikasi berbasis web yang dikembangkan dengan bahasa Java yang berisi berupa class yang digunakan untuk menerima request dan memberi respon melalui protokol http. Servlet memiliki ektensi .java dan merupakan subclass dari HttpServlet. Pada servlet terdapat dua method untuk mengolah request dari client. Kedua method itu adalah \textit{doGet()} dan \textit{doPost()}. \citep{senasama}
\begin{itemize}
	\itemsep0em
	\item Method \textit{doGet()} akan dijalankan jika client mengirimkan HTTP request dengan method GET. Contoh dari method GET, adalah jika user menginputkan sejumlah value pada sebuah tampilan website, maka value-value yang dikirimkan itu akan dimunculkan pada alamat URL web browser.
	\item Method \textit{doPost()} akan dijalankan jika client mengirimkan HTTP response dengan method POST. Contoh dari method POST adalah jika user menginputkan sejumlah value pada sebuah tampilan website, maka value-value yang dikirimkan itu tidak akan dimunculkan pada alamat URL web browser.
\end{itemize}

\section{Google App Engine}
Google App Engine (GAE) adalah layanan untuk mengembangkan dan hosting aplikasi Web di pusat data Google, yang termasuk ke dalam komputasi awan kategori Platform As Services (PaaS). Aplikasi web yang di jalankan pada GAE  dijalankan di beberapa server untuk redundansi dan memungkinkan penskalaan sumber daya sesuai dengan persyaratan trafik tertentu. App Engine secara otomatis akan mengalokasikan sumber daya tambahan ke server untuk mengakomodasi peningkatan beban. Kelebihan dari GAE adalah sebagai berikut: \citep{technopedia}
\begin{itemize}
	\itemsep0em
	\item Server yang tersedia tanpa memerlukan konfigurasi
	\item Power scaling yang menurunkan hingga ke tingkat "gratis" ketika saat penggunaan sumber daya minimal.
	\item Alat komputasi awan yang otomatis
\end{itemize}

\section{Aplikasi Hybrid (\textit{Mobile Hybrid App})}
Aplikasi hybrid adalah aplikasi web yang ditransformasikan menjadi kode native pada platform seperti iOS atau Android. Aplikasi hybrid biasanya menggunakan browser untuk mengijinkan aplikasi web mengakses berbagai fitur di device mobile seperti Push Notification, Contacts, atau Offline Data Storage. Beberapa tools untuk mengembangkan aplikasi hybrid antara lain Phonegap, Rubymotion dan lain-lain \citep{Permana}.
\begin{figure}[H]
	\centering
	\fbox{\includegraphics [width = 12cm, height= 8cm]{gambar/hybrid}}
	\caption{Perbandingan Aplikasi Native \& Hybrid}
	\label{android}
\end{figure}
\par Keuntungan membangun aplikasi hybrid diantaranya pemeliharaan project menjadi semakin mudah jika dibandingkan dengan aplikasi native. Aplikasi hybrid juga, bisa dibangun secara cepat untuk keperluan cross platform dan dana yang bisa menjadi lebih hemat jika dibandingkan dengan native.


\section{Ionic Framework}

Ionic adalah \textit{framework front-end} yang dikhususkan untuk membangun aplikasi \textit{hybrid} dengan HTML5, CSS dan AngularJS. Ionic menggunakan Node.js SASS, AngularJS sebagai \textit{engine}-nya. Ionic dilengkapi dengan komponen-komponen CSS seperti \textit{button, list, card, form, grids, tabs}, dan masih banyak lagi. Jadi Ionic itu merupakan teknologi web yang bisa digunakan untuk membuat suatu aplikasi \textit{mobile}. Karena \textit{hybrid} maka aplikasi hanya dibuat satu kali tetapi sudah bisa dirilis di lebih dari satu \textit{platform} dengan kata lain \textit{cross-platform} \citep{Wahyuni}.


\section{Apache Cordova}
\par Seperti yang telah dijelaskan di atas, bahwa Ionic hanya menyediakan \textit{framework Front-end} sedangkan untuk mengubahnya ke dalam \textit{platform} Android dan IOS, Ionic menggunakan Apache Cordova. Apache Cordova adalah \textit{platform} untuk membangun aplikasi \textit{mobile native} menggunakan HTML, CSS dan JavaScript. Native mobile application yang didukung antara lain Android, iOS, Windows Phone dan Blackberry.
\par Apache Cordova berisi sekumpulan API (\textit{Application Programming Interface}) untuk mengakses device dari perangkat mobile. Device itu antara lain kamera, GPS (\textit{Global Positioning System}), storage dan lain-lain.  Dengan menunggunakan UI (\textit{User Interface}) framework seperti jQuery Mobile, Dojo Mobile atau Sencha Touch, maka kita dapat mengakses API ini. Dengan kata lain kita dapat membangun aplikasi hanya menggunakan HTML, CSS dan Javascript.


\section{Angular}
\par Untuk melakukan implementasi logika, Ionic menggunakan teknologi framework javascript bernama Angular yang menawarkan performa dan respon cepat seperti aplikasi \textit{native}. sebelumnya dikenal dengan nama AngularJS, sekarang dikenal dengan nama Angular (tanpa JS dibelakang). Angular yang merupakan versi terkini dari AngularJS tentu masih banyak peminatnya di dunia pemrograman. Angular telah mengalami banyak sekali perubahan dibandingkan pendahulunya AngularJS. Angular kini sudah side by side dengan framework Javascript modern lainnya seperti React, Vue dan lain-lainnya. Secara konsep Angular sudah lumayan matang, dengan mampu mengakomodir component based dan dengan bergabungnya Typescript milik Microsoft dan RxJS milik ReactiveX untuk mendukung kemapanan framework ini. Performa yang dihasilkan oleh Angular kini bisa disejajarkan dengan para kompetitor dikelasnya.


\section{\textit{Extreme Programming} (XP)}
\par eXtreme Programming adalah sebuah model pengembangan sistem yang menyederhanakan berbagai tahapan proses pengembangan agar tercapainya peningkatan efsiensi dan fleksibilitas sebuah proyek pengembangan perangkat lunak. Model \textit{Extreme Programming} ini mengedepankan proses pengembangan yang lebih responsive terhadap kebutuhan klien dibandingkan dengan model- model tradisional lainnya sambil membangun suatu perangkat lunak dengan kualitas yang lebih baik. \textit{Extreme Programming} menawarkan sebuah dispilin baru dalam pengembangan perangkat lunak secara agile. Nilai dasar yang terkandung di dalam eXtreme Programming adalah : komunikasi (\textit{communication}), kesederhanaan (\textit{simplicity}), umpan balik (\textit{feedback}), keberanian (\textit{courage}) dan kualitas kerja (\textit{quality work}). Menurut Pressman dalam bukunya yang berjudul Sofware Engineering, edisi keenam, proses \textit{Extreme Programming} memiliki kerangka kerja yang terbagi menjadi empat konteks aktivitas utama. Empat konteks tersebut adalah \textit{planning, design, coding dan testing}. Keempat aktivitas inilah yang akan menghasilkan sebuah perangkat lunak yang didasari dengan konsep.
\begin{figure}[H]
	\centering
	\includegraphics [width= 12cm, height= 10cm]{gambar/xp}
	\caption{Aktivitas Utama XP}
	\label{xp}
\end{figure}
Dalam pembuatannya XP memiliki beberapa langkah operasional yang harus dilakukan seperti :

\subsection{\textit{Planning}}
Planning atau perencanaan adalah proses yang dirancang untuk mencapai tujuan tertentu dan pengambilan keputusan untuk mencapai hasil yang diinginkan. Kebutuhan yang dibutuhkan pada tahap ini yaitu:
\begin {itemize}
	\itemsep0em
	\item Teknik pengumpulan data
	\item Analisis kebutuhan sistem
	\item Identifikasi aktor
	\item Identifikasi use case
\end {itemize}
\par Aktifitas planning pada model proses XP berfokus pada mendapatkan gambaran fitur serta fungsi dari perangkat lunak yang akan dibangun. Pada aktivitas ini dimulai dengan membuat kumpulan cerita atau gambaran yang diberikan klien yang kemudian akan menjadi gambaran dasar dari perangkat lunak.
\par Kumpulan tersebut nantinya dikumpulkan dalam sebuah indeks cerita dimana setiap poin dari indeks tersebut ditentukan prioritasnya untuk dibangun. Anggota tim dari pengembang aplikasi nantinya akan menentukan alur pengembangan aplikasi dengan terlebih dahulu memulai mengembangkan tugas dengan resiko dan nilai prioritas yang tinggi terlebih dahulu. Dan selutuh tugas akan selesai dalam tenggat waktu dua minggu.
\par Selama proses pengembangan, klien dapat mengubah, memperkecil, membagi dan membuang setiap rencana dari aplikasi. Tim XP akan mempertimbangkan setiap perubahan yang diajukan client berikutnya akan mengubah setiap rencana dari pengembangan perangkat lunak.

\subsection{\emph{Design}}
\par Aktifitas design dalam pengembangan aplikasi bertujuan untuk mengatur pola logika dalam sistem. Sebuah design yang baik dapat mengurangi ketergantungan antar setiap proses pada sebuah sistem. Dengan begitu, jika salah satu fitur pada sistem mengalami kerusakan, tidak akan mempengaruhi sistem secara keseluruhan.
\par Design pada model proses XP menjadi panduan dalam membangun perangkat lunak yang didasari dari cerita client sebelumnya. Dalam XP, proses design terjadi sebelum dan sesudah aktivitas coding berlangsung. Dimana aktivitas design terjadi secara terus-menerus selama proses pengembangan aplikasi berlangsung.

\subsection{\emph{Coding}}
\par Setelah menyelesaikan pengumpulan cerita dan menyelesaikan \textit{design} untuk aplikasi secara keseluruhan, XP lebih merekomendasikan tim untuk terlebih dahulu membuat modul unit tes yang bertujuan untuk melakukan uji coba setiap cerita yang didapat dari klien. Setelah berbagai unit tes selesai dibangun, barulah tim melanjutkan aktivitasnya ke penulisan coding aplikasi. XP menerapkan konsep \textit{pair programming} dimana setiap tugas sebuah modul dikembangkan oleh dua orang programmer. XP beranggapan, 2 orang akan lebih cepat dan baik dalam menyelesaikan sebuah masalah. Selanjutnya, modul aplikasi yang sudah selesai dibangun akan digabungkan dengan aplikasi utama.

\subsection{\emph{Testing}}
\par Tahapan uji coba pada XP sudah dilakukan juga pada saat tahapan sebelumnya yaitu \textit{coding}. Pengujian perangkat lunak dimaksudkan untuk menguji semua elemen-elemen perangkat lunak yang dibuat apakah sudah sesuai dengan yang diharapkan. XP menerapkan perbaikan masalah kecil dengan sesegera mungkin akan lebih baik dibandingkan menyelesaikan masalah pada saat akan mencapai tenggat akhir. Oleh karena itu, setiap modul yang sedang dikembangkan akan terlebih dahulu mengalami pengujian dengan modul unit tes yang telah dibuat sebelumnya.

\subsection{Keunggulan \textit{Extreme Programming}}
\par Keunggulan dari metode pengembangan perangkat lunak \textit{eXtreme Programming} adalah sebagai berikut: \citep{Michael}
\begin{itemize}
	\itemsep0em
	\item Meningkatkan kepuasan kepada klien.
	\item Membangun system dengan lebih cepat.
	\item Menjalin komunikasi yang baik dengan klien.
	\item Meningkatkan komunikasi dan sifat saling menghargai antar developer.
\end{itemize}

\section {\textit{Test Plan}}
\par Test Plan adalah dokumen yang berisi definisi tujuan dan sasaran pengujian dalam lingkup iterasi (atau proyek), item-item yang menjadi target pengujian, pendekatan yang akan diambil, sumber daya yang dibutuhkan dan point untuk diproduksi. Dengan kata lain test plan dapat disebut sebagai perencanaan atau scenario untuk melakukan testing yang akan dilakukan baik oleh expert atau user umum. \citep{Shanardi}
\par Tujuan umum membuat test plan secara umum adalah untuk memudahkan developer untuk melakukan testing agar testing yang dilakukan menjadi jelas sehingga hasilnya lebih berguna dan efisien. Tentunya dalam proses pembuatan test plan ini memerlukan beberapa langkah seperti berikut :
\begin{itemize}
	\itemsep0em
	\item \textit{Test Plan Identifier}
	\par \textit{Test Plan Identifier} adalah bagian untuk menjelaskan secara singkat mengenai objek yang akan di test. Bisa berupa penjelasan narasi atau berbentuk tabel dengan kategori kategori tertentu. Informasi yang dijelaskan dapat berupa sekilas mengenai subjek testing, nama orang yang bertanggung jawab terhadap testing, penyusun test plan , tanggal dibuat test plan dan tanggal revisi, dan lain-lain.
	
	\item Introduction
	\par Pada bagian introduction dibuat untuk menjelaskan secara narasi, mengenai testing yang akan dilakukan terhadap suatu objek testing. Bagian Introduction dapat dibuat lebih rinci dengan menambahkan sub bab apabila perlu untuk dibuat. Contoh subbab yang dapat dibuat antara lain :
	\begin{itemize}
		\itemsep0em
		\item \textit{Purpose} : untuk menjelaskan tujuan \textit{testing}  secara spesifik.
		\item \textit{Background} : latar belakang mengapa \textit{testing} dilakukan.
		\item \textit{Scope} : sejauh mana \textit{testing} dilakukan
		\item \textit{Definition and acronyms} : Penjelasan mengenai singkatan dan istilah yang ada di dalam dokumen test plan.
	\end{itemize}
	
	\item \textit{Test Items}
	\par Bagian test item menjelaskan mengenai daftar komponen komponen dalam objek testingyang akan di test satu per Satu.
	
	\item \textit{Features to be tested}
	\par Menjelaskan fitur fitur apa saja yang ada di dalam objek \textit{testing}, namun fitur tersebut tidak akan di test pada saat pelaksanaan \textit{testing} dan disertakan penjelasan singkat mengapa fitur tersebut tidak di test pada saat \textit{testing}.
	
	\item \textit{Item pass / fail criteria}
	\par Berisi tentang kriteria-kriteria yang harus dipenuhi sebelum berlanjut ke fase berikutnya contoh :
	\begin{itemize}
		\itemsep0em
		\item Jika suatu item di test sebanyak 10 kali dan 9 kali diantaranya berhasil namun ada 1 dimana benar benar gagal maka \textit{item} tersebut dinyatakan sebagai gagal/ \textit{fail}.
		\item Jika hasil dari suatu item sama dengan hasil yang diharapkan maka \textit{item} tersebut dinyatakan berhasil/\textit{pass}.
		\item \textit{System Crash} akan dinyatakan sebagai \textit{fail},dsb.
	\end{itemize}
	
	
	\item \textit{Testing Task}
	\par Menjelaskan kegiatan testing kepada pihak yang akan melaksanakan kegiatan tersebut.
	
	\item \textit{Responsibilities}
	\par Rincian pihak pihak yang akan bertanggung jawab terhadap suatu kegiatan task di dalam serangkaian kegiatan testing yang akan dilaksanakan
	
	\item \textit{Schedule}
	\par Ada beberapa tujuan dalam membuat schedule di dalam test plan, antara lain :
	\begin{itemize}
		\itemsep0em
		\item Merincikan tolak ukur waktu pengerjaan \textit{testing}.
		\item Estimasi waktu yang dibutuhkan untuk setiap \textit{task}.
		\item Menjadwalkan \textit{testing task} dan \textit{test milestone}.
		\item Merincikan periode pemakaian \textit{testing resource}. 
		\citep{Shanardi}
	\end{itemize}
\end{itemize}

\section {\textit{Usability Testing}}
\par Usability Testing atau tes produk merupakan metode riset untuk mengembangkan dan menyempurnakan produk baru maupun yang telah ada. Inti dari riset ini adalah mendudukkan pengguna/pelanggan di pusat, lalu mengambil pelajaran dari sana . maka dalam test ini, mutlak adanya pengamatan secara langsung \citep{nielsen2012}
\par Dengan mendokumentasikan pengalaman aktual para calon pengguna aplikasi/produk dapat dievealuasi , sebab dari uji ini diharapkan akan menangkap kekuatan dan kelemahan dari setiap aspek yang ada pada aplikasi itu sendiri.

\section {\textit{Unit Testing}}
Unit Testing adalah metode verifikasi perangkat lunak di mana programmer menguji suatu unit program layak untuk tidaknya dipakai. Unit testing ini fokusnya pada verifikasi pada unit yang terkecil pada desain perangkat lunak (komponen atau modul perangkat lunak). Karena dalam sebuah perangkat lunak banyak memiliki unit-unit kecil maka untuk mengujinya biasanya dibuat program kecil atau main program) untuk menguji unit-unit perangkat lunak. Unit-unit kecil ini dapat berupa prosedur atau fungsi, sekumpulan prosedur atau fungsi yang ada dalam satu file jika dalam pemrograman terstruktur, atau kelas, bisa juga kumpulan kelas dalam satu package. \citep{feridi}

%-----------------------------------------------------------------------------%

% Baris ini digunakan untuk membantu dalam melakukan sitasi
% Karena diapit dengan comment, maka baris ini akan diabaikan
% oleh compiler LaTeX.
\begin{comment}
\bibliography{daftar-pustaka}
\end{comment}


%-------------------------------------------------------------------------------
%                            BAB III
%               		METODOLOGI PENELITIAN
%-------------------------------------------------------------------------------

\chapter{METODOLOGI PENELITIAN}

\section{Tempat dan Waktu Penelitian}
\setlength\parindent{30pt} Penelitian ini dilakukan di Jurusan Informatika, Fakultas Matematika dan Ilmu Pengetahuan Alam, Universitas Syiah Kuala Banda Aceh. Waktu yang diperlukan untuk melakukan penelitian ini kurang lebih selama lima bulan, yang dimulai dari bulan Juni 2018 sampai bulan Desember 2018.

% Please remember to add \use{multirow} to your document preamble in order to suppor multirow cells
\begin{table}[H]
	\center
	\caption{Jadwal Penelitian.}
	\label{jadwal}
	\begin{tabular}{|c|l|l|l|l|l|l|l|}
		\hline
		\multirow{2}{*}{No} & \multirow{2}{*}{Keterangan} 	& \multicolumn{6}{c|}{Bulan}           																										\\ \cline{3-8} 
							&                           	& Jul				& Agt  			& Sep			& Okt			& Nov			& Des 					\\ \hline       
		1                   & Studi literatur           	&\cellcolor{gray}	&\cellcolor{gray}	&                   &                   &                   &                       	\\ \hline
		2                   & Penulisan Proposal           	&                   &\cellcolor{gray}	&\cellcolor{gray}	&                   &                   &                        	\\ \hline
		3                   & Pengembangan Aplikasi         &                   &                   & \cellcolor{gray}  & \cellcolor{gray} 	& \cellcolor{gray}  &                            \\ \hline
		4                   & Evaluasi Sistem               &                   &                   &           		&             		&\cellcolor{gray}	&  \cellcolor{gray}    \\ \hline
		5                   & Penulisan Laporan Akhir       &                   &                   &                   &        			&      				& \cellcolor{gray}     \\ \hline
	\end{tabular}
\end{table}

\section{Alat dan Bahan}
Alat yang digunakan pada penelitian ini adalah sebagai berikut :

\begin{enumerate}[a.]
\item Perangkat Keras (\textit{Hardware})
	\begin{itemize}
		\item 1 unit Laptop ASUS ROG Intel(R) Core(TM) i7 6700HQ
		\item RAM 8 GB DDR3
		\item Harddisk 1 TB
		\item 1 unit Android Samsung S7
	\end{itemize}

\item Perangkat Lunak (\textit{Software})
	\begin{itemize}
		\item Visual Studio Code
		\item Cordova 7.0.0
		\item Ionic Framework 4
		\item SDK Android 27.0.1
	\end{itemize}
\end{enumerate}

\section{Metode Penelitian}
Skema dari alur tahapan penelitian dapat dilihat pada Gambar \ref{alur}
\vspace{-0.4cm}
\begin{figure}[H]
	\center
	\includegraphics [width = 6cm, height= 10cm]{gambar/alur2}
	\caption{Diagram Alir Penelitian}
	\label{alur}
\end{figure}

\subsection{Analisis Permasalahan}
Aplikasi ini merupakan aplikasi berbasis mobile yang berguna untuk membantu pangkalan gas LPG Subsidi 3 Kg dalam melakukan pelaporan penyaluran tabung gas subsidi. Aplikasi ini juga membantu agen gas LPG selaku distributor dan pengawasan dari pangkalan gas untuk melakukan rekap laporan penyaluran tabung gas 3 Kg. Aplikasi ini melakukan pelaporan penyaluran tabung gas subsidi dengan menggunakan NIK(Nomor Induk Kependudukan) dan no telepon sebagai data acuan yang valid. Setelah itu data tersebut langsung di kirim langsung ke agen gas LPG bersangkutan. Adapun fitur-fitur yang terdapat pada aplikasi ini adalah sebagai berikut
\begin{itemize}
		\itemsep0em
		\item Masuk ke dalam aplikasi.
		\item Mencatat penjualan tabung gas menggunakan NIK sebagai acuan.
		\item Melihat histori pembelian gas konsumen.
		\item Melihat foto konsumen yang membeli gas
		\item Mengubah profile biodata user.
\end{itemize}

\subsection{Studi Literatur}
Studi literatur dilakukan dengan mencari jurnal baik nasional maupun internasional, buku, serta beberapa literatur elektronik yang diunduh dari internet yang terkait dengan penelitian ini. Studi literatur juga diperoleh dengan meneliti aplikasi atau perangkat lunak yang berkaitan dengan penelitian. Studi literatur digunakan sebagai bahan referensi selama proses penelitian.

\subsection{Pengumpulan Data}
Data yang digunakan dalam penelitian ini adalah data penyaluran tabung ke masyarakat dan format pelaporan yang dipakai oleh pangkalan. Data tersebut didapatkan dari agen dan pangkalan yang menjadi objek penelitian. Pada tahap pengembangan ini, akan menghasilkan luaran-luaran seperti :
	\begin{itemize}
		\item Menjelaskan lebih detail mengenai bagaimana alur penyaluran gas LPG 3 Kg dari agen gas LPG sampai ke konsumen.
		\item Menjelaskan secara detail bagaimana format pelaporan yang dipakai agen gas untuk memantau penyaluran gas LPG 3 Kg.
	\end{itemize}

\subsection{Perancangan dan Pembuatan Sistem}
Dalam merancang aplikasi ini, digunakan metode pengembangan perangkat lunak yaitu Scrum. Metode Scrum sangat fleksibel terhadap perubahan-perubahan sehingga cocok digunakan pada aplikasi ini. Berikut merupakan tahapan perancangan sistem yang akan dilakukan:

\begin{enumerate}[1.]
	\item \emph {Planning}
	
	Pada tahap ini dilakukan analisa terhadap masalah yang terjadi di lapangan dalam penelitian ini yaitu pangkalan gas LPG 3 Kg dan mencari solusi yang akan memecahkan masalah tersebut dengan mengembangkan suatu sistem. Setelah itu dilakukan tahap inisiasi awal pengembangan sistem yaitu melakukan analisis kebutuhan pada setiap pengguna sistem yaitu Agen dan Pangkalan Gas LPG yang dilanjutkan dengan membuat \textit{storyboard} yaitu gambaran yang menceritakan jalannya sistem yang dibuat secara singkat. Tahap ini dianggap telah selesai apabila semua kebutuhan fungsional dari aplikasi telah didapatkan.
	
	

	\item \textit{Design}
	
	\par Tahap ini fokus pada desain pembuatan perangkat lunak termasuk struktur data, arsitektur perangkat lunak, representasi antarmuka dan prosedur pengkodean. Proses tahap ini dilakukan dengan terlebih dahulu mengidentifikasi persona, \textit{value proposition canvas} (vpc), membuat use case diagram untuk setiap aktor dan storyboard dan juga membuat \textit{deployment diagram} seperti pada gambar \ref{deployment}. Proses tersebut dilakukan untuk memastikan setiap desain dan cara kerja aplikasi yang dibangun dapat digunakan oleh masing-masing kelompok pengguna.
	\par Perancangan sistem terbagi 2 yaitu aplikasi \textit{mobile web} yang akan digunakan oleh Pangkalan Gas LPG untuk melakukan pencatatan penyaluran tabung gas dan \textit{web admin} yang akan digunakan oleh Agen Gas LPG untuk melakukan rekapitulasi data penyaluran. Tahap ini dianggap telah selesai apabila telah menghasilkan sebuah desain rancangan aplikasi dalam bentuk use-case diagram dan deployment diagram, dan model diagram lainnya.
	

	
	\item \textit{Coding}
	
	Pada tahapan ini, dilakukan implementasi terhadap desain yang telah dibuat sebelumnya ke dalam bentuk program. Penulisan kode program dalam penelitian ini menggunakan Ionic Framework yang ditulis dalam bahasa pemrograman Typescript untuk aplikasi berbasis Android dan untuk web service yang akan digunakan oleh aplikasi untuk berkomunikasi dengan database akan menggunakan Google App Engine yang ditulis dalam bahasa pemrograman Java. Tahap ini dianggap telah selesai apabila semua kebutuhan fungsional dan non-fungsional telah diselesaikan.

	 \item \textit{Testing}
	
	Tahapan \textit{testing} yang dilakukan diantaranya adalah sebagai berikut:
	\begin{enumerate}[a.]
			\itemsep0em
			\item \textit{Usability}
			\newline Pengujian \textit{usability} digunakan untuk mengetahui seberapa mudah aplikasi dapat dijalankan oleh pengguna. Teknik pengujian \textit{usability} yang digunakan adalah dengan memberikan kuesioner kepada setiap kelompok user yaitu agen gas LPG dan pangkalan gas LPG. Jenis pertanyaan yang ada pada kuesioner mengacu pada kuesioner SUS (\textit{System Usability Scale}). Dalam pengujian \textit{usability} ini, melibatkan sekitar 8 pengguna yang berbeda. SUS terdiri dari 10
			pertanyaan dengan menggunakan skala dari 1 sampai 5. Skor akhir dari pengujian. SUS akan berada pada kisaran 0-100 \%. Berdasarkan skor akhir SUS tersebut dapat diketahui seberapa tinggi tingkat usability dan akseptabilitas desain sistem aplikasi yang dikembangkan. Berdasarkan skor akhir SUS tersebut dapat diketahui seberapa tinggi tingkat usability dan interpretasi desain sistem aplikasi yang dikembangkan. Berikut
			nilai interpretasi yang digunakan :
			
								\begin{table}[H]
								\center
								\caption{Skor Interpretasi.}
								\label{nilaiInterpretasi}
								\begin{tabular}{|l|l|}
									\cline{1-2}
									\multicolumn{2}{|c|}{\textbf{Interpretasi Skor}}                                                        \\ \cline{1-2}
									\multirow{2}{*}{\textbf{Persentase Pencapaian \%}} & \multirow{2}{*}{\textbf{Interpretasi}}          \\
									&                                               \\ \hline
									0-20                                                    &  Sangat tidak layak                           \\ \hline
									21-40                                                   &  Tidak layak                                  \\ \hline                                          
									41-60                                                   &  Cukup layak                                  \\ \hline                                          
									61-80                                                   &  Layak                                  \\ \hline                                          
									81-100                                                  &  Sangat layak                                  \\ \hline                                          
								\end{tabular}
							\end{table}
			
			\item \textit{Unit Testing}
			\newline Unit testing fokus pada verifikasi pada unit yang terkecil pada desain perangkat lunak (komponen atau modul perangkat lunak). Karena dalam sebuah perangkat lunak banyak memiliki unit-unit kecil maka untuk mengujinya biasanya dibuat program kecil atau main program) untuk menguji unit-unit perangkat lunak.
		
	\end{enumerate}
	
	\par Tahap ini dapat dilakukan setelah ataupun berdampingan dengan tahap \textit{coding}. Tahap ini dianggap telah selesai apabila semua pengujian yang telah dikerjakan menghasilkan nilai dengan skala tertentu
	
	
\end{enumerate}


% Baris ini digunakan untuk membantu dalam melakukan sitasi
% Karena diapit dengan comment, maka baris ini akan diabaikan
% oleh compiler LaTeX.
\begin{comment}
\bibliography{daftar-pustaka}
\end{comment}


%-------------------------------------------------------------------------------
%                            BAB IV
%               		HASIL DAN PEMBAHASAN
%-------------------------------------------------------------------------------

\chapter{HASIL DAN PEMBAHASAN}
	\section{Analisis Kebutuhan}
	
	Hasil dari analisis kebutuhan yang telah dilakukan adalah mendapatkan persona, storyboard dan use case diagram untuk masing-masing pengguna.
	
	\subsection{Kelompok Pengguna}
	Kelompok pengguna dari aplikasi ini telah dapat diidentifikasikan pada tahap analisis kebutuhan pada sistem. terdapat 2 kelompok pengguna yang menggunakan aplikasi ini:
		 \begin{enumerate}[1.]
		 	\item Agen
		 		\newline Pengguna yang menggunakan aplikasi berbasis web untuk mengelola rencana penerimaan pasokan tabung ke pangkalan, melakukan rekapitulasi data penyaluran tabung pada pangkalan.
		 	\item Pangkalan
		 		\newline Pengguna yang menggunakan aplikasi berbasis android untuk melakukan pencatatan penjualan tabung per hari, mendaftarkan pelanggan baru, melakukan verifikasi pada penerimaan pasokan tabung.
		 \end{enumerate}
	 
	 Persona untuk masing-masing kelompok pengguna dapat dilihat pada lampiran 1
	
	\subsection{Storyboard}
	Dengan menggunakan \textit{Storyboard} kita dapat memetakan secara visual kegiatan pengguna sebelum dan sesudah menggunakan aplikasi. Berdasarkan hasil pengamatan dan analisis kebutuhan dapat digambarkan \textit{Storyboard} seperti berikut:
	
	\vspace{-0.4cm}
	\begin{figure}[H]
		\center
		\includegraphics [width = 14cm]{gambar/storyboard/storyboard-1-(old)}
		\caption{Kegiatan Sebelum Menggunakan Aplikasi}
		\label{storyboardOld1}
	\end{figure}
	
	\begin{figure}[H]
		\center
		\includegraphics [width = 14cm]{gambar/storyboard/storyboard-1-(new)}
		\caption{Kegiatan Sesudah Menggunakan Aplikasi}
		\label{storyboardNew1}
	\end{figure}
	
	
	
	\subsection{\textit{Value Proposition Canvas }(VPC)}
	Dengan Menggunakan \textit{Value Proposition Canvas} (VPC) ini, kita dapat menguji apakah desain aplikasi dan fitur-fiturnya telah memenuhi kebutuhan dari pengguna aplikasi. Dari hasil pengamatan yang telah dilakukan maka dapat digambarkan VPC sebagai berikut:
	
		\vspace{-0.4cm}
	\begin{figure}[H]
		\center
		\includegraphics [width = 14cm]{gambar/model/VPC-Gas}
		\caption{Kegiatan Sebelum Menggunakan Aplikasi}
		\label{vpc}
	\end{figure}

	
	\subsection{\textit{Use Case}}
	\textit{Use-Case Diagram} digunakan untuk memetakan fungsionalitas aplikasi pada masing-masing pengguna. Dari hasil pengamatan dapat digambarkan \textit{Use-Case Diagram} sebagai berikut:
	
	\vspace{-0.4cm}
	\begin{figure}[H]
		\center
		\includegraphics [width = 14cm]{gambar/model/use-case-diagram}
		\caption{Use-Case Diagram}
		\label{usecase}
	\end{figure}
	
	\section{Desain dan Pembuatan Sistem}
	
	\subsection{Desain Sistem}
	\par Desain Sistem adalah proses desain pada aplikasi yang akan dibangun. proses dilakukan berdasarkan proses analisis kebutuhan aplikasi yang telah dilakukan pada proses sebelumnya. proses ini terdiri dari beberapa tahap yaitu seperti perancangan \textit{deployment diagram},
	perancangan arsitektur aplikasi dalam bentuk \textit{component diagram}, perancangan struktur basis data dalam bentuk \textit{class diagram}, dan juga perancangan tampilan antar muka dari aplikasi yang akan dibangun.
	
	\par tahap pertama adalah merancang \textit{deployment diagram}, dimana \textit{deployment diagram} adalah diagram yang menjelaskan bagaimana sistem aplikasi akan bekerja dan digunakan oleh pengguna. aplikasi yang dibangun terdiri dari 2 platform yang berbeda dimana pengguna pangkalan akan menggunakan platform android untuk melakukan pencatatan penjualan tabung, dan lain-lain, sedangkan pengguna agen akan menggunakan platform \textit{web-based} untuk rekapitulasi laporan penyaluran. Berikut rancangan \textit{deployment diagram} yang telah dibuat:
	
		\vspace{-0.4cm}
	\begin{figure}[H]
		\center
		\includegraphics [width = 14cm]{gambar/model/deployment}
		\caption{Diagram Deployment}
		\label{deployment}
	\end{figure}

	\par tahap selanjutnya adalah merancang \textit{component diagram}, dimana \textit{component diagram} adalah diagram yang menjelaskan arsitektur sistem aplikasi yang akan dibangun secara keseluruhan. Berikut rancangan \textit{component diagram} yang telah dibuat:
	
	\vspace{-0.4cm}
	\begin{figure}[H]
		\center
		\includegraphics [width = 14cm]{gambar/model/component}
		\caption{Diagram Component}
		\label{component}
	\end{figure}

	\pagebreak
	\par Dilanjutkan dengan tahap perancangan model basis data menggunakan \textit{class diagram}. diagram ini dibuat berdasarkan fungsionalitas dari sistem yang akan dibangun. Berikut rancangan \textit{class diagram} yang telah dibuat:
	
		\vspace{-0.4cm}
	\begin{figure}[H]
		\center
		\includegraphics [width = 14cm]{gambar/model/class-diagram}
		\caption{Diagram Kelas Model Basis Data }
		\label{class}
	\end{figure}

	\par Selanjutnya adalah perancangan layanan web, perancangan layanan web disini memiliki peran yang sangat penting dan vital, dimana layanan web harus dapat menangani setiap proses transaksi data yang dilakukan oleh sistem. layanan web ini menggunakan RESTful API untuk dapat berinteraksi dengan \textit{platform} android dan aplikasi lain melalui HTTP URL.
	
	\par Untuk perancangan antar muka dari aplikasi ini menggunakan ionic framework karena menyediakan komponen-komponen antar muka pengguna seperti \textit{tabs, card, dan button} yang cocok untuk digunakan pada platform \textit{mobile}.
	
	\par Tampilan Aplikasi berbasis android untuk pangkalan gas LPG 3Kg ini terdiri 3 halaman utama yaitu halaman register pelanggan baru, halaman pembelian tabung, dan halaman penerimaan tabung. Pertama-tama untuk masuk ke dalam aplikasi, pangkalan harus melakukan \textit{login} menggunakan no HP yang telah terdaftar seperti pada gambar \ref{tampilanLoginPangkalan}. setelah itu akan muncul sms no HP tersebut berisi sebuah kode OTP(\textit{One Time Password}) yang digunakan untuk verifikasi no HP valid atau tidak. setelah proses \textit{login} berhasil maka dilanjut ke halaman beranda yang berisi menu utama aplikasi seperti pada Gambar \ref{tampilanBerandaPangkalan}.
	\par Pada halaman registrasi pelanggan baru terdapat \textit{form} mengenai data pelanggan setelah itu dilanjutkan pengambilan foto ktp dan foto diri pelanggan seperti pada Gambar \ref{tampilanRegisterPangkalan}. Pada saat pembelian tabung, pangkalan harus mencari pelanggan terlebih dahulu menggunakan NIK (Nomor Induk Kependudukan) seperti pada Gambar \ref{tampilanPembelianPangkalan} setelah itu maka akan muncul halaman detail pelanggan berisi detail data pelanggan tersebut. pada halaman ini terdapat 2 tombol aksi yaitu riwayat pembelian dan beli tabung seperti pada Gambar \ref{tampilanPembelianPangkalan}. setelah pangkalan memilih beli tabung selanjutnya pangkalan harus memilih jumlah tabung yang dibeli dan dilanjutkan pengambilan foto pembeli sebagai bukti pembelian seperti pada Gambar \ref{tampilanPembelianPangkalan}. Pada saat penerimaan pasokan tabung, pangkalan harus melakukan verifikasi penerimaan tabung pada menu penerimaan tabung seperti pada Gambar \ref{tampilanPenerimaanPangkalan}.
	
	\vspace{-0.2cm}
	\begin{figure}[H]
		\center
		\includegraphics [width = 7cm]{gambar/android/login}
		\vspace{1cm}
		\includegraphics [width = 7cm]{gambar/android/verify}
		\caption{Tampilan Login aplikasi}
		\label{tampilanLoginPangkalan}
	\end{figure}
	
	\begin{figure}[H]
		\center
		\includegraphics [width = 7cm]{gambar/android/beranda}
		\caption{Tampilan Halaman Beranda}
		\label{tampilanBerandaPangkalan}
	\end{figure}

	\begin{figure}[H]
		\center
		\includegraphics [width = 6cm]{gambar/android/register}
		\caption{Tampilan Halaman Register Pelanggan}
		\label{tampilanRegisterPangkalan}
	\end{figure}

	\begin{figure}[H]
		\center
		\includegraphics [width = 6cm]{gambar/android/penerimaan}
		\caption{Tampilan Halaman Penerimaan Pasokan Tabung}
		\label{tampilanPenerimaanPangkalan}
	\end{figure}

	\begin{figure}[H]
		\center
		\includegraphics [width = 6cm]{gambar/android/cari-pelanggan}
		\vspace{1cm}
		\hspace{1cm}
		\includegraphics [width = 6cm]{gambar/android/detail-pelanggan}
		\includegraphics [width = 6cm]{gambar/android/pilih-tabung}
		\hspace{1cm}
		\includegraphics [width = 6cm]{gambar/android/bukti-pembelian}
		\caption{Tampilan Halaman Pembelian Tabung}
		\label{tampilanPembelianPangkalan}
	\end{figure}

	\par Tampilan Aplikasi berbasis web untuk agen gas LPG 3Kg ini terdiri dari 5 halaman utama yaitu halaman beranda, halaman pangkalan, halaman rencana pasokan tabung, halaman pembeli, dan halaman untuk melakukan ekspor laporan penyaluran.
	\par Sebelumnya masuk ke dalam aplikasi ini, agen harus melakukan login menggunakan email khususnya email google seperti pada Gambar \ref{tampilanLoginAgen}. Setelah berhasil melakukan login maka akan ditujukan ke halaman beranda, pada halaman ini terdapat status jumlah pangkalan, jumlah penjualan rata-rata bulan ini dan tahun ini seperti pada \ref{tampilanBerandaAgen}.
	\par Halaman pangkalan berfungsi untuk menampilkan daftar pangkalan pada agen tersebut, disini agen dapat melakukan penambahan dengan menekan tombol "tambah pangkalan"dan perubahan pada data pangkalan dengan menekan tombol "profil" seperti pada Gambar \ref{tampilanDaftarPangkalanAgen}. Pada halaman ini juga memiliki aksi untuk membuat rencana pasokan tabung untuk pangkalan dengan menekan tombol "rencana pasokan" yang akan ditujukan ke halaman daftar rencana pasokan tabung untuk pangkalan tersebut seperti pada Gambar \ref{tampilanPenerimaanAgen}. Halaman rencana pasokan ini berfungsi untuk menginput data rencana pasokan untuk bulan depan. Data-data ini nantinya harus di verifikasi oleh pangkalan via aplikasi berbasis android sehingga datanya akan tersinkronisasi.
	\par Halaman laporan berfungsi untuk melakukan ekspor laporan penyaluran ke dalam bentuk PDF seperti pada Gambar \ref{tampilanLaporanAgen}.
	
	\begin{figure}[H]
		\center
		\includegraphics [width = 9cm]{gambar/web/login}
		\caption{Tampilan Halaman Login untuk Agen Gas LPG 3Kg}
		\label{tampilanLoginAgen}
	\end{figure}
	
	\begin{figure}[H]
		\center
		\includegraphics [width = 9cm]{gambar/web/beranda}
		\caption{Tampilan Halaman Beranda untuk Agen Gas LPG 3Kg}
		\label{tampilanBerandaAgen}
	\end{figure}

	\begin{figure}[H]
		\center
		\includegraphics [width = 9cm]{gambar/web/pangkalan}
		\caption{Tampilan Halaman Daftar Pangkalan }
		\label{tampilanDaftarPangkalanAgen}
	\end{figure}

	\begin{figure}[H]
		\center
		\includegraphics [width = 9cm]{gambar/web/penerimaan}
		\caption{Tampilan Halaman Rencana Penerimaan Pasokan Tabung}
		\label{tampilanPenerimaanAgen}
	\end{figure}

	\begin{figure}[H]
		\center
		\includegraphics [width = 9cm]{gambar/web/Laporan}
		\caption{Tampilan Halaman Ekspor Laporan}
		\label{tampilanLaporanAgen}
	\end{figure}
	
	
	
	\subsection{Pembuatan Sistem}
	
	\begin{enumerate}[a.]
		\item Layanan Web
		\\ Layanan web dikembangkan menggunakan bahasa pemrograman java dengan memanfaatkan pustaka Google Endpoints API untuk membangun sebuah layanan web yang memiliki prinsip RESTful API. dan juga menggunakan Google Datastore sebagai media penyimpanan basis data. Google Datastore merupakan layanan basis data NoSQL yang dibuat untuk penskalaan otomatis, kinerja tinggi, dan kemudahan dalam pengembangan aplikasi. Dikarenakan kita menggunakan Google Datastore sebagai Layanan penyimpanan basis data, kita akan menggunakan salah satu pustaka Java yang dimilikinya yaitu Objectify. Pada pembuatan layanan web (\textit{webservice}) hanya membutuhkan \textit{Model} Basis Data, \textit{Controller}, dan API. Pada pembuatan \textit{Model} basis data, desain basis datanya akan mengikuti dari desain \textit{class diagram} yang dibuat sebelumnya. tahap pembuatan layanan web yang pertama adalah membuat kelas \textit{Model} untuk masing-masing entitas seperti pada Gambar \ref{modelWebservice}. Setelah itu membuat kelas \textit{controller} yang akan menangani setiap operasi untuk memanipulasi data didalam \textit{Model} seperti pada Gambar \ref{controllerWebservice}. Tahap selanjutnya adalah membuat kelas API yang akan menangani pemanggilan setiap \textit{endpoint} yang layanan web seperti pada Gambar \ref{apiWebservice}. Setelah semua komponen telah selesai, maka akan dilakukan pengujian terhadap layanan web untuk mengetahui apakah setiap endpoint berjalan dengan seharusnya seperti pada Gambar \ref{pengujianApi}.
		
			\lstset{language=Java,
			basicstyle=\ttfamily\scriptsize\color{black},
			keywordstyle=\color{javapurple}\bfseries,
			stringstyle=\color{javared},
			commentstyle=\color{javagreen},
			morecomment=[s][\color{javadocblue}]{/**}{*/},
			numbers=left,
			numberstyle=\tiny\color{black},
			showstringspaces=false,
			numbersep=10pt,
			tabsize=4,
			showspaces=false,
			showstringspaces=false,
			autogobble=true,
			xleftmargin=2em
		}
	
		\begin{lstlisting}[caption=Potongan kode \textit{model} basis data layanan web, label=modelWebservice]
		@Entity
		public class Penerimaan {
		public Penerimaan(Date tanggal, Integer jmlTabung, Ref<Laporan> laporan, Ref<Pangkalan> pangkalan, Boolean statusVerifikasi) {
		setTanggal(tanggal);
		setJmlTabung(jmlTabung);
		setLaporan(laporan);
		setPangkalan(pangkalan);
		setStatusVerifikasi(statusVerifikasi);
		}
		.......................................................................
		@SuppressWarnings("unused")
		private Penerimaan(){}
		.......................................................................
		/**
		* kembalikan key dari pengguna.
		* @return key dari laporan atau null jika entitas belum tersimpan
		*/
		public Key<Penerimaan> getKey() {
		if(this.id == null)
		return null;
		
		return Key.create(Penerimaan.class, this.id);
		}
		......................................................................
		@Id 
		private Long id;
		public Long getId() {
		return id;
		}
		......................................................................
		@Index
		private Date tanggal;
		public Date getTanggal() {
		return tanggal;
		}
		public void setTanggal(Date tanggal) {
		if(tanggal == null)
		throw new NullPointerException("Tanggal tidak boleh Null");
		
		this.tanggal = tanggal;
		}
		......................................................................
		private Integer jmlTabung;
		public Integer getJmlTabung() {
		return jmlTabung;
		}
		public void setJmlTabung(Integer jmlTabung) {
		if(jmlTabung == null)
		throw new NullPointerException("Jml Tabung tidak boleh Null");
		
		if(jmlTabung < 0)
		throw new NullPointerException("Jml Tabung tidak boleh Minus");
		
		this.jmlTabung = jmlTabung;
		}
		
		\end{lstlisting}	
		
		
		\begin{lstlisting}[caption=Potongan kode \textit{controller} layanan web, label=controllerWebservice]
		public class PenerimaanAtur {
			/**
			* Tambah {@link Penerimaan} baru.
			* 
			* @param tanggal Tanggal {@link Penerimaan} baru.
			* @param jmlTabung Jumlah Tabung dari {@link Penerimaan} baru.
			* @param idPangkalan dari {@link Penerimaan} baru.
			* @param idPelanggan Id Pelanggan dari {@link Penerimaan} baru.
			*  
			* @throws TidakDitemukanException 
			* @throws StokTidakCukupException 
			*/
			
			public Penerimaan baru(Date tanggal, Integer jmlTabung, Long idPangkalan) throws EntitasDuplikasiException, TidakDitemukanException, StokTidakCukupException {
			Pangkalan pangkalan;
			
			
			// Pastikan id pelanggan ada di entity penerimaan
			try {
			pangkalan = new PangkalanAtur().cari(idPangkalan);
			} catch (TidakDitemukanException e) {
			throw new TidakDitemukanException("Tidak ditemukan Pangkalan dengan id: '"+idPangkalan+"'");
			}
			
			// Pastikan penerimaan tidak memiliki tanggal dan pangkalan yang sama
			try {
			new PenerimaanAtur().cariTanggalPangkalan(tanggal, pangkalan);
			throw new EntitasDuplikasiException("telah ada penjualan dengan tanggal :"+tanggal+"dan pangkalan : "+pangkalan.getNamaPangkalan());
			} catch (TidakDitemukanException e) {}
			
			
			Ref<Pangkalan> refPangkalan = Ref.create(pangkalan);
			
			//tambah penerimaan dengan refLaporan null
			Penerimaan penerimaan = new Penerimaan(tanggal, jmlTabung, null, refPangkalan, false);
			ofy().save().entity(penerimaan).now();
			
			
			return penerimaan;
			}
			
			/**
			* Cari {@link Penerimaan} dengan memakai key-nya.
			* 
			* @param key ID dari {@link Penerimaan} yang hendak dicari.
			* 
			* @return {@link Penerimaan} dengan key yang diberikan.
			* 
			* @throws TidakDitemukanException Jika {@link Penerimaan} dengan key yang diberikan tidak ditemukan.
			*/
			
			public Penerimaan cari(Key<Penerimaan> key) throws TidakDitemukanException {
			if (key == null)
			throw new NullPointerException("Key dari Penerimaan yang hendak dicari tidak boleh null.");
			
			Penerimaan penerimaan = ofy().load().key(key).now();
			if (penerimaan == null)
			throw new TidakDitemukanException("Tidak ada penerimaan dengan id: '" + key.getId() + "'!");
			
			return penerimaan;
			}
		}
		
		\end{lstlisting}
		
			\begin{lstlisting}[caption=Potongan kode \textit{controller} layanan web, label=controllerWebservice]
		public class PenerimaanAtur {
		/**
		* Tambah {@link Penerimaan} baru.
		* 
		* @param tanggal Tanggal {@link Penerimaan} baru.
		* @param jmlTabung Jumlah Tabung dari {@link Penerimaan} baru.
		* @param idPangkalan dari {@link Penerimaan} baru.
		* @param idPelanggan Id Pelanggan dari {@link Penerimaan} baru.
		*  
		* @throws TidakDitemukanException 
		* @throws StokTidakCukupException 
		*/
		
		public Penerimaan baru(Date tanggal, Integer jmlTabung, Long idPangkalan) throws EntitasDuplikasiException, TidakDitemukanException, StokTidakCukupException {
		Pangkalan pangkalan;
		
		
		// Pastikan id pelanggan ada di entity penerimaan
		try {
		pangkalan = new PangkalanAtur().cari(idPangkalan);
		} catch (TidakDitemukanException e) {
		throw new TidakDitemukanException("Tidak ditemukan Pangkalan dengan id: '"+idPangkalan+"'");
		}
		
		// Pastikan penerimaan tidak memiliki tanggal dan pangkalan yang sama
		try {
		new PenerimaanAtur().cariTanggalPangkalan(tanggal, pangkalan);
		throw new EntitasDuplikasiException("telah ada penjualan dengan tanggal :"+tanggal+"dan pangkalan : "+pangkalan.getNamaPangkalan());
		} catch (TidakDitemukanException e) {}
		
		
		Ref<Pangkalan> refPangkalan = Ref.create(pangkalan);
		
		//tambah penerimaan dengan refLaporan null
		Penerimaan penerimaan = new Penerimaan(tanggal, jmlTabung, null, refPangkalan, false);
		ofy().save().entity(penerimaan).now();
		
		
		return penerimaan;
		}
		
		/**
		* Cari {@link Penerimaan} dengan memakai key-nya.
		* 
		* @param key ID dari {@link Penerimaan} yang hendak dicari.
		* 
		* @return {@link Penerimaan} dengan key yang diberikan.
		* 
		* @throws TidakDitemukanException Jika {@link Penerimaan} dengan key yang diberikan tidak ditemukan.
		*/
		
		public Penerimaan cari(Key<Penerimaan> key) throws TidakDitemukanException {
		if (key == null)
		throw new NullPointerException("Key dari Penerimaan yang hendak dicari tidak boleh null.");
		
		Penerimaan penerimaan = ofy().load().key(key).now();
		if (penerimaan == null)
		throw new TidakDitemukanException("Tidak ada penerimaan dengan id: '" + key.getId() + "'!");
		
		return penerimaan;
		}
		}
		
		\end{lstlisting}	
		
			\begin{lstlisting}[caption=Potongan kode API layanan web, label=apiWebservice]
		@Api(name="penerimaan",
		version="v1",
		title = "Penerimaan",
		description="API untuk Model Penerimaan. ",
		apiKeyRequired=AnnotationBoolean.TRUE)
		public class PenerimaanApi {
		@ApiMethod(name = "daftar", httpMethod = HttpMethod.GET)
		public List<PenerimaanJSON> daftar() {
		
		List<Penerimaan> daftar = new PenerimaanAtur().daftar();
		List<PenerimaanJSON> daftarJSON = new ArrayList<PenerimaanJSON>();
		
		for(Penerimaan baris : daftar)
		daftarJSON.add(new PenerimaanJSON(baris));
		
		return daftarJSON;
		}
		
		@ApiMethod(name = "daftarByPangkalan", httpMethod = HttpMethod.POST)
		public List<PenerimaanJSON> daftarByPangkalan(DaftarByPangkalanJSON params) throws TidakDitemukanException {
		Pangkalan pangkalan = new PangkalanAtur().cari(params.idPangkalan);
		List<Penerimaan> daftar = new PenerimaanAtur().daftarByPangkalan(pangkalan);
		List<PenerimaanJSON> daftarJSON = new ArrayList<PenerimaanJSON>();
		
		for(Penerimaan baris : daftar) {
		daftarJSON.add(new PenerimaanJSON(baris));
		}
		
		return daftarJSON;
		}
		
		@ApiMethod(name = "detail", httpMethod = HttpMethod.GET)
		public PenerimaanJSON detail(@Named("idPenerimaan") Long idPenerimaan) throws BadRequestException, TidakDitemukanException {
		
		Penerimaan penerimaan = new PenerimaanAtur().cari(idPenerimaan);
		
		if(penerimaan==null || penerimaan.getKey() == null) {
		throw new TidakDitemukanException("Tidak ada pangkalan dengan id: '" + idPenerimaan + "'!");
		}
		
		return new PenerimaanJSON(penerimaan);
		}
		}
		
		\end{lstlisting}
	
		\begin{figure}[H]
			\center
			\includegraphics [width = 12cm]{gambar/pengujianApi}
			\caption{Contoh Pengujian Endpoint API pada layanan web (\textit{Webservice})}
			\label{pengujianApi}
		\end{figure} 
		
		\item Aplikasi Mobile
		\\ Aplikasi berbasis \textit{android} dikembangkan menggunakan Ionic Framework yang menggunakan teknologi HTML, JavaScript dan CSS sebagai tampilan \textit{frontend} aplikasi. Pengembangan aplikasi ini mengikuti salah satu prinsip pembuatan aplikasi secara umum yaitu MVC (Model View Controller). Model (providers) berisi fungsi untuk masing-masing method HTTP yang dipanggil oleh aplikasi dari layanan web. \textit{Model} dibuat menggunakan Typescript. \textit{Model} berperan penting pada proses penarikan dan penginputan data ke Google Datastore. Contoh Kode model pada ionic dapat dilihat pada Gambar \ref{modelMobile}. \textit{View} berfungsi untuk mengatur tampilan aplikasi yang ditampilkan pada pengguna melalui HTML dan CSS. Contoh Kode view pada ionic dapat dilihat pada Gambar \ref{modelView}. \textit{Controller} untuk mengatur halaman yang ditampilkan kepada pengguna dan pemanggilan API. \\
		
		\vspace{-0.4cm}
		
		\lstset{language=HTML,
			basicstyle=\ttfamily\scriptsize\color{black},
			keywordstyle=\color{javapurple}\bfseries,
			stringstyle=\color{javared},
			commentstyle=\color{javagreen},
			morecomment=[s][\color{javadocblue}]{/**}{*/},
			numbers=left,
			numberstyle=\tiny\color{black},
			showstringspaces=false,
			numbersep=10pt,
			tabsize=4,
			showspaces=false,
			showstringspaces=false,
			autogobble=true,
			xleftmargin=2em
		}
	
		\begin{lstlisting}[caption=Potongan kode \textit{model} aplikasi berbasis android, label=modelMobile]
		export class RestService {
		apiUrl = 'https://gasapp-2018.appspot.com/_ah/api';
		
		constructor(public http: HttpClient) {
		console.log('Hello RestProvider Provider');
		}
		
		userAuthentication(phoneNumber) {
		return new Promise((resolve, reject) => {
		let headers = new HttpHeaders()
		let data = {noHp:phoneNumber};
		this.http.post(this.apiUrl+'/pangkalan/v1/auth', JSON.stringify(data), {headers: headers})
		.subscribe(res => {
		resolve(res);
		}, (err) => {
		reject(err);
		});
		});
		}
		
		addPelanggan(data){
		return new Promise((resolve, reject) => {
		let headers = new HttpHeaders()
		
		this.http.post(this.apiUrl+'/pelanggan/v1/tambah', JSON.stringify(data), {headers: headers})
		.subscribe(res => {
		resolve(res);
		}, (err) => {
		reject(err);
		});
		});
		}
		
		getPelanggan(nik, idPangkalan) {
		return new Promise((resolve, reject) => {
		let headers = new HttpHeaders()
		let data = {nik:nik, idPangkalan:idPangkalan};
		
		this.http.post(this.apiUrl+'/pelanggan/v1/cariNik', JSON.stringify(data), {headers: headers}).subscribe(data => {
		resolve(data);
		}, (err) => {
		console.log(err);
		reject(err);
		});
		});
		}
		}
		
		\end{lstlisting}
		
		\begin{lstlisting}[caption=Potongan kode \textit{view} aplikasi berbasis android, label=viewMobile]
			<ion-header translucent="true">
			<ion-toolbar id="toolbar-home" style="padding-top:0">
			<ion-title>
			<ion-img [src]="logo" class="logo"></ion-img>
			</ion-title>
			<ion-buttons slot="start">
			<ion-menu-button autoHide="false"></ion-menu-button>
			</ion-buttons>
			</ion-toolbar>
			</ion-header>
			
			<ion-content fullscreen ion-padding color="light">
			
			<!-- <ion-slides pager=true [options]="slideOptions"> 
			<ion-slide *ngFor="let slide of slides">
			<ion-img [src]="slide" class="slide-image"></ion-img>
			</ion-slide> 
			</ion-slides> -->
			<div class="wrap-cover parallax-obj">
			<div class="bg-profile shadow-3">
			<ion-row class="panel-stats">
			<ion-col id="stok" size="6">
			<ion-text color="light" class="stats">
			<h4>Stok Tabung</h4>
			<p>{{statistik.stokTabung}}</p>
			</ion-text> 
			</ion-col>
			<ion-col id="hari" size="6">
			<ion-text color="light" class="stats">
			<h4>Terjual Hari Ini</h4>
			<p>{{statistik.hariIni}}</p>
			</ion-text> 
			</ion-col>
			<ion-col id="bulan" size="6">
			<ion-text color="light" class="stats">
			<h4>Terjual Bulan Ini</h4>
			<p>{{statistik.bulanIni}}</p>
			</ion-text> 
			</ion-col>
			<ion-col id="tahun" size="6">
			<ion-text color="light" class="stats">
			<h4>Terjual Tahun Ini</h4>
			<p>{{statistik.tahunIni}}</p>
			</ion-text> 
			</ion-col>
			</ion-row>
			</div>
			</div>
		
		\end{lstlisting}
		
		\begin{lstlisting}[caption=Potongan kode \textit{controller} aplikasi berbasis android, label=controllerMobile]
		export class HomePage {
		logo = 'assets/img/logoPutih.png';
		
		userData:any;
		statistik = {
		stokTabung: 0,
		hariIni: 0,
		bulanIni: 0,
		tahunIni: 0
		}
		
		constructor(
		private route:ActivatedRoute,
		private custom:CustomService,
		private menu: MenuController,
		private storage: Storage,
		private restService: RestService
		) {
		if(this.route.snapshot.paramMap.has('flashdata')){
		this.custom.showAlert("", this.route.snapshot.paramMap.get('flashdata'));
		}
		
		this.storage.get("userData").then(res => {
		this.userData = res;
		
		this.restService.statistikTabung(this.userData.idPangkalan).then(res => {
		let data:any = res;
		this.statistik.stokTabung = data.stokTabung;
		this.statistik.hariIni = data.tabungTerjualHariIni;
		this.statistik.bulanIni = data.tabungTerjualBulanIni;
		this.statistik.tahunIni = data.tabungTerjualTahunIni;
		})
		console.log(this.userData);
		this.menu.enable(true, 'menu');
		this.menu.open('menu');
		
		}).catch(err => {
		console.log(err);
		});
		
		}
		
		ngOnInit() {
		
		}
		
		slideOptions = {
		autoplay: {
		delay: 5000,
		}
		};
		}
		
		\end{lstlisting}
		
		 aplikasi ini juga menggunakan layanan Google Firebase untuk dapat melakukan proses login menggunakan nomor HP atau dinamakan dengan \textit{Phone Authentication} seperti pada Gambar \ref{firebaseAuth}
		
		\begin{figure}[H]
			\center
			\includegraphics [width = 12cm]{gambar/firebaseAuth}
			\caption{Halaman Konsole  \textit{Phone Authentication} pada Google Firebase}
			\label{firebaseAuth}
		\end{figure} 
	
		\item Aplikasi Web
		\\ Aplikasi berbasis \textit{web} dikembangkan menggunakan Bahasa pemrograman Java dengan menggunakan prinsip yang sama yaitu MVC (Model View Controller). dimana \textit{model} menggunakan pustaka HTTP untuk memanggil method HTTP pada layanan web dan pustaka GSON untuk melakukan proses konversi dari JSON ke dalam objek java. untuk \textit{view} menggunakan JSP (\textit{Java Server Page}) untuk menampilkan isi halaman \textit{web} yang akan dibangun. JSP memiliki struktur bahasa yang hampir sama dengan HTML (\textit{Hypertext Markup Language}). untuk \textit{controller} menggunakan pustaka HTTP Servlet untuk menangani permintaan HTTP URL dan mengatur pemanggilan \textit{view dan model}. Aplikasi ini nanti akan berjalan pada platform Google App Engine (GAE)\\
		
		\vspace{-0.4cm}
		
			\lstset{language=Java,
			basicstyle=\ttfamily\scriptsize\color{black},
			keywordstyle=\color{javapurple}\bfseries,
			stringstyle=\color{javared},
			commentstyle=\color{javagreen},
			morecomment=[s][\color{javadocblue}]{/**}{*/},
			numbers=left,
			numberstyle=\tiny\color{black},
			showstringspaces=false,
			numbersep=10pt,
			tabsize=4,
			showspaces=false,
			showstringspaces=false,
			autogobble=true,
			xleftmargin=2em
		}
	
			\begin{lstlisting}[caption=Potongan kode \textit{model} aplikasi berbasis web, label=modelWeb]
			public class PangkalanService extends AppService{
				
				private static final String baseUrl=AppService.getWebService();
				
				public ListPangkalanJson getPangkalan()
				{
				String restUrl = baseUrl+"/pangkalan/v1/daftar";
				
				String jsonString = getJsonString(restUrl);
				ListPangkalanJson jsonObject = null;
				
				//Konversi json string ke class object
				try{
				Gson gson = new GsonBuilder().create();
				jsonObject = gson.fromJson(jsonString, ListPangkalanJson.class);            
				}
				catch(JsonSyntaxException ex)
				{
				ex.printStackTrace();
				}
				
				return jsonObject;
				}
				
				public PangkalanJson getDetailPangkalan(Long idPangkalan)
				{
				String restUrl = baseUrl+"/pangkalan/v1/detail/"+idPangkalan;
				
				String jsonString = getJsonString(restUrl);
				PangkalanJson jsonObject = null;
				
				//Konversi json string ke class object
				try{
				Gson gson = new GsonBuilder().create();
				jsonObject = gson.fromJson(jsonString, PangkalanJson.class);            
				}
				catch(JsonSyntaxException ex)
				{
				ex.printStackTrace();
				}
				
				return jsonObject;
				}
		}
		
		\end{lstlisting}
		
			\lstset{language=HTML,
			basicstyle=\ttfamily\scriptsize\color{black},
			keywordstyle=\color{javapurple}\bfseries,
			stringstyle=\color{javared},
			commentstyle=\color{javagreen},
			morecomment=[s][\color{javadocblue}]{/**}{*/},
			numbers=left,
			numberstyle=\tiny\color{black},
			showstringspaces=false,
			numbersep=10pt,
			tabsize=4,
			showspaces=false,
			showstringspaces=false,
			autogobble=true,
			xleftmargin=2em
		}
			\begin{lstlisting}[caption=Potongan kode \textit{view} aplikasi berbasis web, label=viewWeb]
		 <%@ taglib prefix="c" uri="http://java.sun.com/jsp/jstl/core" %>
		<!-- Main content -->
		<section class="content">
		<!-- Small boxes (Stat box) -->
		<div class="row">
		<c:choose>
		<c:when test="${sessionScope.userGroup == 'administrator'}">
		<div class="col-lg-6 col-xs-6">
		<!-- small box -->
		<div class="small-box bg-aqua">
		<div class="inner">
		<h3>${jmlAgen}</h3>
		<p>Jumlah Agen</p>
		</div>
		<div class="icon">
		<i class="fa fa-user"></i>
		</div>
		<a href="#" class="small-box-footer">More info <i class="fa fa-arrow-circle-right"></i></a>
		</div>
		</div><!-- ./col -->
		<div class="col-lg-6 col-xs-6">
		<!-- small box -->
		<div class="small-box bg-green">
		<div class="inner">
		<h3>${jmlPangkalan}</h3>
		<p>Jumlah Pangkalan</p>
		</div>
		<div class="icon">
		<i class="fa fa-user"></i>
		</div>
		<a href="#" class="small-box-footer">More info <i class="fa fa-arrow-circle-right"></i></a>
		</div>
		</div><!-- ./col -->
		</c:when>   
		
		\end{lstlisting}
		
		
			\lstset{language=Java,
			basicstyle=\ttfamily\scriptsize\color{black},
			keywordstyle=\color{javapurple}\bfseries,
			stringstyle=\color{javared},
			commentstyle=\color{javagreen},
			morecomment=[s][\color{javadocblue}]{/**}{*/},
			numbers=left,
			numberstyle=\tiny\color{black},
			showstringspaces=false,
			numbersep=10pt,
			tabsize=4,
			showspaces=false,
			showstringspaces=false,
			autogobble=true,
			xleftmargin=2em
		}
			\begin{lstlisting}[caption=Potongan kode \textit{controller} aplikasi berbasis web, label=controllerWeb]
				@SuppressWarnings("serial")
				@WebServlet(
				name="Dashboard",
				urlPatterns={"/dashboard"}, 
				description="Halaman Dashboard (admin)"
				)
				public class Dashboard extends HttpServlet{
				
				
				@Override
				protected void doGet(HttpServletRequest req, HttpServletResponse resp) throws ServletException, IOException {
				
				HttpSession session = req.getSession();
				//cek otentikasi user dengan session
				if (session == null || session.getAttribute("userMail") == null) {
				resp.sendRedirect(AppService.getBaseUrl(req)+"/login");
				return;
				}
				
				List<AgenJson> agenJsons;
				List<PangkalanJson> pangkalanJsons;
				
				
				if(session.getAttribute("userGroup").toString().compareTo("agen") == 0) {
				Long idAgen =  (Long) session.getAttribute("userIdAgen");
				
				pangkalanJsons = new PangkalanService().getPangkalanByAgen(idAgen).getItems();
				
				StatistikPenjualanParam param = new StatistikPenjualanParam(idAgen);
				StatistikJson statistikAgen =  new StatistikService().getStatistik(param);
				
				req.setAttribute("rataBulanan", statistikAgen.getRataPenjualanBulanIni());
				req.setAttribute("rataTahunan", statistikAgen.getRataPenjualanTahunIni());
				}else {
				agenJsons = new StatistikService().getAgen().getItems();
				pangkalanJsons = new StatistikService().getPangkalan().getItems();
				req.setAttribute("jmlAgen",agenJsons.size());
				}
				
				
				
				req.setAttribute("baseUrl", AppService.getBaseUrl(req));
				req.setAttribute("pageTitle", "Dashboard");
				req.setAttribute("webserviceUrl", AppService.getWebService());
				req.setAttribute("content", "Home");
				
				req.setAttribute("jmlPangkalan", pangkalanJsons.size());
				
				
				resp.setContentType("text/html");
				RequestDispatcher jsp = req.getRequestDispatcher(KonstantaWebApp.JSP_FOLDER+"Dashboard.jsp");
				jsp.forward(req, resp);
				}
				
				
				}
		
		\end{lstlisting}
	
		Aplikasi berbasis web ini menggunakan layanan Google Sign-in untuk melakukan proses login dengan memakai email google sebagai ID Pengenal. Untuk proses implementasinya hanya dengan menanamkan \textit{script} javascript pada halaman login seperti pada Gambar \ref{loginWeb}. Aplikasi ini juga memakai pustaka java yang bernama itext untuk melakukan ekspor data penyaluran tabung dalam bentuk \textit{file} PDF. Berikut potongan kode untuk melakukan proses ekspor pada Gambar \ref{eksporWeb}.
	
			\lstset{language=HTML,
			basicstyle=\ttfamily\scriptsize\color{black},
			keywordstyle=\color{javapurple}\bfseries,
			stringstyle=\color{javared},
			commentstyle=\color{javagreen},
			morecomment=[s][\color{javadocblue}]{/**}{*/},
			numbers=left,
			numberstyle=\tiny\color{black},
			showstringspaces=false,
			numbersep=10pt,
			tabsize=4,
			showspaces=false,
			showstringspaces=false,
			autogobble=true,
			xleftmargin=2em
		}
	
		\begin{lstlisting}[caption=Potongan kode Login memakai email aplikasi berbasis web, label=loginWeb]
			<script src="assets/modules/bootstrap/js/bootstrap.min.js"></script>
			<script type="text/javascript">
			var BASE_URL = '${baseUrl}';
			$(document).ready(function(){
			$('[data-toggle="tooltip"]').tooltip();
			});
			</script>
		
		\end{lstlisting}
	
		\lstset{language=Java,
		basicstyle=\ttfamily\scriptsize\color{black},
		keywordstyle=\color{javapurple}\bfseries,
		stringstyle=\color{javared},
		commentstyle=\color{javagreen},
		morecomment=[s][\color{javadocblue}]{/**}{*/},
		numbers=left,
		numberstyle=\tiny\color{black},
		showstringspaces=false,
		numbersep=10pt,
		tabsize=4,
		showspaces=false,
		showstringspaces=false,
		autogobble=true,
		xleftmargin=2em
	}
		\begin{lstlisting}[caption=Potongan kode ekspor data aplikasi berbasis web, label=eksporWeb]
		protected void generatePdf(Document doc, String[] periode, EksporPelaporanJson data) 
		throws IOException {
		doc.add(new Paragraph("Data Laporan Penyaluran Gas 3Kg" ).setBold().setFontSize(14).setTextAlignment(TextAlignment.CENTER));
		Table detailPangkalan = new Table(3);
		detailPangkalan.setMarginTop(18);
		detailPangkalan.setMarginLeft(8);
		
		detailPangkalan.addCell(new Cell().add(new Paragraph("Nama Pangkalan")).setBorder(Border.NO_BORDER));
		detailPangkalan.addCell(new Cell().add(new Paragraph(":")).setBorder(Border.NO_BORDER));
		detailPangkalan.addCell(new Cell().add(new Paragraph(data.getNamaPangkalan())).setBorder(Border.NO_BORDER));
		
		detailPangkalan.addCell(new Cell().add(new Paragraph("Nama Pemilik")).setBorder(Border.NO_BORDER));
		detailPangkalan.addCell(new Cell().add(new Paragraph(":")).setBorder(Border.NO_BORDER));
		detailPangkalan.addCell(new Cell().add(new Paragraph(data.getNamaPemilik())).setBorder(Border.NO_BORDER));
		
		detailPangkalan.addCell(new Cell().add(new Paragraph("Alamat")).setBorder(Border.NO_BORDER));
		detailPangkalan.addCell(new Cell().add(new Paragraph(":")).setBorder(Border.NO_BORDER));
		detailPangkalan.addCell(new Cell().add(new Paragraph(data.getAlamatPangkalan())).setBorder(Border.NO_BORDER));
		
		detailPangkalan.addCell(new Cell().add(new Paragraph("Agen Penyuplai")).setBorder(Border.NO_BORDER));
		detailPangkalan.addCell(new Cell().add(new Paragraph(":")).setBorder(Border.NO_BORDER));
		detailPangkalan.addCell(new Cell().add(new Paragraph(data.getNamaAgen())).setBorder(Border.NO_BORDER));
		
		detailPangkalan.addCell(new Cell().add(new Paragraph("Bulan")).setBorder(Border.NO_BORDER));
		detailPangkalan.addCell(new Cell().add(new Paragraph(":")).setBorder(Border.NO_BORDER));
		detailPangkalan.addCell(new Cell().add(new Paragraph(periode[0]+" "+periode[1])).setBorder(Border.NO_BORDER));
		
		doc.add(detailPangkalan);
		doc.add(generateTable(periode, data.getPelaporan()));
		}
		\end{lstlisting}

	\end{enumerate}

	
	\section{Pengujian Sistem}
	
		\subsection{Metode \textit{Whitebox}}
			\par Pada pengujian \textit{whitebox}, aplikasi ini menggunakan JUnit yang dapat melakukan pengujian secara otomatis. Aspek-aspek yang diuji pada pengujian ini adalah bagaimana program yang dibangun bukan hanya dapat berjalan dengan semestinya (\textit{test success}), tapi pada saat diberikan sebuah parameter yang salah program dapat mengembalikan error yang semestinya (\textit{test fail}). Aplikasi yang dibangun telah dianggap lulus pengujian apabila mampu melewati semua \textit{test} secara sempurna 100\%. 
			
				\lstset{language=Java,
				basicstyle=\ttfamily\scriptsize\color{black},
				keywordstyle=\color{javapurple}\bfseries,
				stringstyle=\color{javared},
				commentstyle=\color{javagreen},
				morecomment=[s][\color{javadocblue}]{/**}{*/},
				numbers=left,
				numberstyle=\tiny\color{black},
				showstringspaces=false,
				numbersep=10pt,
				tabsize=4,
				showspaces=false,
				showstringspaces=false,
				autogobble=true,
				xleftmargin=2em
			}
			
			\begin{lstlisting}[caption=Potongan kode Pengujian JUnit dengan kasus berhasil (\textit{test success}), label=testSuccess]
				//cek tambah penerimaan
				@Test
				public void baru() throws EntitasDuplikasiException,
				TidakDitemukanException, StokTidakCukupException {
				Agen agen = this.buatAgen();
				Pangkalan pangkalan = this.buatPangkalan(agen);
				
				Penerimaan penerimaan = new PenerimaanAtur().baru(tanggalPenerimaan1, jmlTabungPenerimaan1, pangkalan.getId());
				assertNotNull(penerimaan);
				assertNotNull(penerimaan.getId());
				assertNotNull(penerimaan.getKey());
				assertNotNull(penerimaan.getTanggal());
				assertEquals(penerimaan.getTanggal().compareTo(tanggalPenerimaan1), 0);
				assertNotNull(penerimaan.getJmlTabung());
				assertEquals(penerimaan.getJmlTabung().compareTo(jmlTabungPenerimaan1), 0);
				assertNotNull(penerimaan.getPangkalan());
				assertEquals(penerimaan.getPangkalan().compareTo(Ref.create(pangkalan)), 0);
				assertNull(penerimaan.getLaporan());
				assertNotNull(penerimaan.getStatusVerifikasi());
				assertEquals(penerimaan.getStatusVerifikasi().compareTo(false), 0);
				}
			\end{lstlisting}
			
				\begin{lstlisting}[caption=Potongan kode Pengujian JUnit dengan kasus gagal (\textit{test fail}), label=testFail]
			//cek tambah penerimaan dengan tanggal null
			@Test(expected = NullPointerException.class)
			public void baruTanggalNull() throws EntitasDuplikasiException, TidakDitemukanException, StokTidakCukupException {
			Agen agen = this.buatAgen();
			Pangkalan pangkalan = this.buatPangkalan(agen);
			
			new PenerimaanAtur().baru(null, jmlTabungPenerimaan1, pangkalan.getId());
			}
			\end{lstlisting}
		
			\begin{figure}[H]
				\center
				\includegraphics [width = 14cm]{gambar/kode/junit-success}
				\caption{Pengujian JUnit Berhasil}
				\label{junitSuccess}
			\end{figure}
		
			\begin{figure}[H]
				\center
				\includegraphics [width = 14cm]{gambar/kode/junit-fail}
				\caption{Pengujian JUnit Gagal}
				\label{junitFail}
			\end{figure}
	
	
	\subsection{Metode \textit{Blackbox}}
	
	Pada pengujian menggunakan metode \textit{blackbox}, aspek yang diuji adalah aspek fungsionalitas dari aplikasi tanpa melihat proses logika yang terjadi dibelakang. Berikut hasil dari pengujian ini dapat dilihat pada tabel \ref{ujiWeb} untuk aplikasi berbasis \textit{web} dan tabel \ref{ujiMobile} untuk aplikasi android.
	
	\begin{longtable}{ |c|p{3cm}|p{3cm}|p{3cm}|p{2cm}|}
	\caption{Pengujian \textit{blackbox} pada aplikasi berbasis web}
	\label{ujiWeb} \\ \hline
	\textbf{No.}                  &  \textbf{Nama Pengujian}         & \textbf{Skenario}                                       & \textbf{Luaran}              & \textbf{Hasil Pengujian} \\ \hline
	
	
	\multirow{2}{*}{1.}  & 	\multirow{2}{*}{Login Akun} & Klik \textit{link} "Login menggunakan Email Google"           & Tampil form login google yang harus diisi oleh pengguna.             & Berhasil \\ \cline{3-5}
	& & Masukkan email google dan passwordnya           & muncul halaman beranda         & Berhasil \\ \hline
	
	\multirow{3}{*}{2.}  & 	\multirow{3}{*}{Tambah Pangkalan} & Klik menu pangkalan           & Tampil daftar pangkalan yang ada.             & Berhasil \\ \cline{3-5}
	& & Klik tombol "tambah pangkalan"           & Tampil form untuk menginput data pangkalan baru.             & Berhasil \\ \hline
	& & Masukkan data pangkalan ke dalam form dan klik tombol "simpan"           & Muncul peringatan bahwa pangkalan telah berhasil ditambah.             & Berhasil \\ \hline
	
	\multirow{4}{*}{3.}  & 	\multirow{4}{*}{Edit Pangkalan} & Klik menu pangkalan           & Tampil daftar pangkalan yang ada.             & Berhasil \\ \cline{3-5}
	& & Klik tombol "profil"           & Tampil form data pangkalan yang ter-\textit{disable} semua.             & Berhasil \\ \cline{3-5}
	& & Klik tombol "edit"           & form tadi dapat diedit             & Berhasil \\ \cline{3-5}
	& & Edit data pangkalan pada form dan klik tombol "simpan"           & Muncul peringatan bahwa pangkalan telah berhasil diedit.             & Berhasil \\ \hline
	
	
	\multirow{4}{*}{4.}  & 	\multirow{4}{*}{\parbox{3cm}{\centering Tambah Rencana Pasokan Tabung}} & Klik menu pangkalan           & Tampil daftar pangkalan yang ada.             & Berhasil \\ \cline{3-5}
	& & Klik tombol "Rencana Pasokan"           & Tampil halaman daftar pasokan tabung untuk pangkalan tersebut.             & Berhasil \\ \cline{3-5}
	& & Klik tombol "tambah penerimaan"           & Muncul form untuk menginput data rencana pasokan.             & Berhasil \\ \cline{3-5}
	& & Masukkan data pasokan tabung ke dalam form dan klik tombol "simpan"     & Muncul peringatan bahwa data pasokan telah berhasil ditambah.             & Berhasil \\ \hline
	
	\multirow{2}{*}{5.}  & 	\multirow{2}{*}{Ekspor Laporan} & Klik menu data laporan          & Tampil Form Ekspor Laporan.             & Berhasil \\ \hline
	& & Masukkan periode laporan dan milik pangkalan mana setelah itu klik tombol "Ekspor Data"     & Muncul laporan penyaluran tabung dalam bentuk file PDF.             & Berhasil \\ \hline
		
	\end{longtable}
	
	\begin{longtable}{ |c|p{3cm}|p{3cm}|p{3cm}|p{2cm}|}
		\caption{Pengujian \textit{blackbox} pada aplikasi berbasis android}
		\label{ujiMobile} \\ \hline
		\textbf{No.}                  &  \textbf{Nama Pengujian}         & \textbf{Skenario}                                       & \textbf{Luaran}              & \textbf{Hasil Pengujian} \\ \hline
		
		
		\multirow{2}{*}{1.}  & 	\multirow{2}{*}{Login Akun} & Masukkan no HP akun pangkalan yang telah terdaftar           & masuk SMS kode OTP dan  Tampil halaman verifikasi OTP         & Berhasil \\ \cline{3-5}
		& & Masukkan Kode OTP dan tekan tombol "verifikasi"           & muncul halaman beranda         & Berhasil \\ \hline
		
		\multirow{5}{*}{2.}  & 	\multirow{5}{*}{\parbox{3cm}{\centering Register Pelanggan}} & Klik menu "register pelanggan"           & Tampil form untuk menginput data pelanggan            & Berhasil \\ \cline{3-5}
		& & Masukkan data pelanggan ke dalam form dan klik tombol "simpan"           & Tampil halaman untuk mengupload berkas pelanggan.             & Berhasil \\ \cline{3-5}
		& &  Tekan tombol "foto diri" untuk mengambil foto diri pelanggan        & Tombol "foto diri" akan terdapat logo centang             & Berhasil \\ \cline{3-5}
		& &  Tekan tombol "foto KTP" untuk mengambil foto ktp pelanggan        & Tombol "foto KTP" akan terdapat logo centang             & Berhasil \\ \hline
		& &  Tekan tombol "simpan" untuk menginput data pelanggan baru   & Muncul peringatan bahwa data pelanggan baru telah berhasil ditambah.             & Berhasil \\ \hline
		
		\multirow{1}{*}{3.}  & 	\multirow{1}{*}{Penjualan Tabung} & Klik menu "Penjualan Tabung"          & Tampil halaman untuk mencari pelanggan.             & Berhasil \\ \cline{3-5}
		\multirow{5}{*}{} & \multirow{5}{*}{} & Masukkan nik dari pelanggan yang membeli tabung dan tekan nama pelanggan           & Tampil data detail pelanggan.             & Berhasil \\ \cline{3-5}
		& & Klik tombol "beli tabung"           & tampil halaman pilihan jumlah tabung & Berhasil
		 \\ \cline{3-5}
		& & tekan salah pilihan tabung           & Tampil halaman untuk mengupload bukti pembelian.             & Berhasil \\ \cline{3-5}
		& &  Tekan tombol "foto Bukti Pembelian" untuk mengambil foto Bukti Pembelian pelanggan        & Tombol "foto Bukti Pembelian" akan terdapat logo centang             & Berhasil \\ \cline{3-5}
		& &  Tekan tombol "simpan" untuk menginput data penjualan  & Muncul peringatan bahwa penjualan baru telah berhasil.             & Berhasil \\ \hline
		
		\multirow{1}{*}{4.}  & 	\multirow{1}{*}{\parbox{3cm}{\centering Riwayat Pembelian Tabung}} & Klik menu "Penjualan Tabung"          & Tampil halaman untuk mencari pelanggan.             & Berhasil \\ \hline
		 \multirow{3}{*}{} & \multirow{3}{*}{} & Masukkan nik dari pelanggan yang membeli tabung dan tekan nama pelanggan           & Tampil data detail pelanggan.             & Berhasil \\ \cline{3-5}
		& & Tekan tombol "riwayat pembelian"           & tampil halaman daftar riwayat pembelian tabung & Berhasil
		\\ \cline{3-5}
		& & tekan salah satu dari daftar pembelian           & Tampil halaman yang menampilkan foto bukti pembelian.             & Berhasil \\ \hline
	
		
		
		\multirow{3}{*}{5.}  & 	\multirow{3}{*}{\parbox{3cm}{\centering Verifikasi Penerimaan Pasokan Tabung}} & Klik menu "Penerimaan Tabung"           & Tampil daftar penerimaan pasokan yang ada.             & Berhasil \\ \cline{3-5}
		& & geser salah salah satu daftar penerimaan dan tekan tombol centang           & Tampil peringatan konfirmasi penerimaan tabung.             & Berhasil \\ \cline{3-5}
		& & Tekan tombol "ya"           & Muncul peringatan bahwa penerimaan tabung telah berhasil dan jumlah tabung bertambah.             & Berhasil \\ \hline
		
	\end{longtable}

	Berdasarkan tabel \ref{ujiWeb} dan tabel \ref{ujiMobile} seluruh pengujian yang telah dilakukan berhasil sehingga aplikasi web dan android sudah dapat berjalan dengan baik. 
	
	\subsection{Metode \textit{Usability Testing}}
	
	Setelah melakukan pengujian sistem, selanjutnya aplikasi akan diuji dengan pengujian \textit{Usability Testing} pada delapan responden dan mengisi kuesioner SUS. SUS memiliki kuesioner yang terdiri dari 10 pertanyaan. Contoh dari SUS dapat dilihat pada lampiran 2 dan \textit{Test Plan} pengujian ini dapat dilihat pada tabel \ref{test plan}.
	
		\begin{center}
		\begin{table}[H]
			\center
			\caption{\textit{Test Plan }pengujian \textit{usability} aplikasi}
			\label{test plan}
			\begin{tabular}{ |p{12cm}|  }
				\hline
				\multicolumn{1}{|c|}{\textbf{\textit{Test Plan} \textit{Usablity Testing}}} \\
				\hline
				Lokasi :
				\begin{enumerate}
					\item PT Pasha Jaya, Meuraxa
					\item Starjazz Kupi, Batoh
					\item Dhapu Kupi, Batoh
					\item Pangkalan Gas Ateuk Meunjeng
					\item Pangkalan Gas Lingke
					\item Pangkalan Gas Lampeunereut
				\end{enumerate}
				Skenario :
				\begin{enumerate}[a.]
					\item Mobile:
						\begin{enumerate}[1.]
							\item Pengguna \textit{login} kedalam aplikasi
							\item Pengguna mendaftarkan pelanggan baru
							\item Pengguna mencatat pembelian tabung 
							\item Pengguna konfirmasi penerimaan pasokan tabung
							\item Pengguna keluar dari aplikasi
							
						\end{enumerate}
					\item Web:
						\begin{enumerate}[1.]
							\item Pengguna \textit{login} kedalam aplikasi
							\item Pengguna menambahkan pangkalan baru.
							\item Pengguna menginput rencana pasokan tabung untuk pangkalan.
							\item Pengguna mencetak laporan penyaluran tabung.
							\item Pengguna keluar dari aplikasi
							
						\end{enumerate}
				\end{enumerate}
				 \\
				\hline
			\end{tabular}
		\end{table}
	\end{center}

	\begin{center}
	\begin{tabular}{ |p{12cm}|  }
		\hline
		Alat   : \textit{Smartphone} Android untuk menggunakan aplikasi dan laptop untuk menggunakan aplikasi berbasis web\\
		Hasil : Hasil Pengujian SUS terdapat pada Tabel \ref{hasilSusAgen} dan Tabel \ref{hasilSusPangkalan}\\
		\hline
	\end{tabular}
\end{center}

\begin{table}[H]
	\center
	\caption{Hasil Pengujian SUS pada Aplikasi Berbasis Web untuk Agen.}
	\label{hasilSusAgen}
	\begin{tabular}{|c|l|l|l|l|l|l|l|l|l|l|l|}
		\hline
		\multirow{2}{*}{Nama} & \multicolumn{10}{c|}{Skor Untuk Tiap Pertanyaan} &  \multirow{2}{0.5cm}{Jml} \\ \cline{2-11} 
		&1 &2  &3 &4 &5 &6 &7 &8 &9 &10& \\
		\hline
		User 1 &8 &8 &8 &6 &8 &10 &10 &6 &10 &6 &80 \\ 
		\hline
		User 2 &8 &6 &8 &8 &8 &8 &10 &8 &6 &6 &76 \\ 
		\hline
		User 3 &8 &10 &10 &4 &8 &10 &10 &10 &10 &8 &88 \\ 
		\hline
		User 4 &8 &6 &10 &8 &8 &8 &10 &8 &8 &8 &82 \\ 
		\hline
		User 5 &8 &6 &8 &8 &6 &8 &8 &6 &8 &8 &74 \\ 
		\hline
		User 6 &6 &8 &6 &8 &8 &8 &8 &6 &8 &8 &74 \\ 
		\hline
		User 7 &6 &6 &8 &8 &10 &6 &6 &10 &8 &8 &76 \\ 
		\hline
		User 8 &8 &8 &10 &6 &10 &10 &8 &6 &6 &6 &80 \\ 
		\hline
	\end{tabular}
\end{table}

\begin{table}[H]
	\center
	\caption{Hasil Pengujian SUS pada Aplikasi Berbasis Android untuk Pangkalan.}
	\label{hasilSusPangkalan}
	\begin{tabular}{|c|l|l|l|l|l|l|l|l|l|l|l|}
		\hline
		\multirow{2}{*}{Nama} & \multicolumn{10}{c|}{Skor Untuk Tiap Pertanyaan} &  \multirow{2}{0.5cm}{Jml} \\ \cline{2-11} 
		&1 &2  &3 &4 &5 &6 &7 &8 &9 &10& \\
		\hline
		User 1 &10 &6 &10 &8 &10 &10 &10 &8 &10 &6 &88 \\ 
		\hline
		User 2 &8 &6 &8 &8 &4 &8 &8 &8 &8 &6 &72 \\ 
		\hline
		User 3 &8 &10 &8 &6 &10 &10 &10 &10 &8 &6 &86 \\ 
		\hline
		User 4 &10 &8 &8 &6 &8 &8 &8 &10 &8 &6 &80 \\ 
		\hline
		User 5 &8 &6 &6 &8 &8 &8 &8 &8 &6 &8 &74 \\ 
		\hline
		User 6 &6 &6 &8 &10 &8 &8 &10 &8 &8 &6 &78 \\ 
		\hline
		User 7 &8 &10 &6 &6 &10 &6 &6 &8 &8 &6 &74 \\ 
		\hline
		User 8 &8 &8 &8 &6 &6 &6 &6 &8 &8 &6 &70 \\ 
		\hline
	\end{tabular}
\end{table}

Dari hasil pengujian usability diperoleh skor yang dihitung dari nilai rata-rata seluruh responden untuk aplikasi berbasis web untuk Agen yaitu sebesar 78 dan aplikasi berbasis android untuk Pangkalan sebesar 77. Maka untuk mendapatkan interpretasi skor tersebut, dihitung persentase pencapaian dengan rumus berikut. 

\[persentase = \frac{nilai Total}{nilai Maximum}\times100\]
\[persentase Aplikasi Web = \frac{78}{100}\times100=78\%\]
\[persentase Aplikasi Android = \frac{77}{100}\times100=77\%\]

Persentase pencapaian untuk aplikasi ini yaitu 78\% untuk aplikasi web dan 77\% untuk aplikasi android. Kemudian dilakukan komparasi nilai persentase pencapaian dengan interpretasi skor. Aplikasi-aplikasi ini berada pada rentang 61-80\% dengan interpretasi skor “Layak”. Berdasarkan hasil tersebut, dapat disimpulkan bahwa aplikasi-aplikasi ini memiliki fitur yang baik dan sesuai dengan kebutuhan setiap kelompok pengguna serta dapat digunakan untuk membantu dalam hal laporan penyaluran gas LPG 3 Kg.  

Selama pengujian usability berlangsung dilakukan pula penghitungan terhadap waktu pengerjaan (\textit{time on task}) dan kesalahan pengguna dalam menyelesaikan tugas-tugas \textit{(task error)} yang diberikan. Penghitungan\textit{ \textit{time on task}} dilakukan untuk mengetahui berapa lama waktu yang dibutuhkan pengguna untuk mengerjakan tugas yang diberikan dan diukur dalam satuan detik (s). Hasil dari penghitungan \textit{time on task} untuk aplikasi untuk agen dapat dilihat pada tabel \ref{timeOnTaskAgen} dan untuk pangkalan dapat dilihat pada tabel \ref{timeOnTaskPangkalan}. 

\begin{table}[H]
	\center
	\caption{\textit{Time On Task} Pengujian Aplikasi Berbasis Web untuk Agen.}
	\label{timeOnTaskAgen}
	\begin{tabular}{|c|l|l|l|l|l|}
		\hline
		\multirow{2}{*}{Nama} & \multicolumn{5}{c|}{Tugas (s)} \\ \cline{2-6} 
		&1 &2  &3 &4 &5  \\
		\hline
		User 1 &10 &20 &27 &15 &6  \\ 
		\hline
		User 2 &12 &22 &29 &16 &5  \\ 
		\hline
		User 3 &8 &19 &28 &15 &5  \\ 
		\hline
		User 4 &9 &20 &23 &18 &7  \\ 
		\hline
		User 5 &8 &19 &24 &17 &8  \\ 
		\hline
		User 6 &10 &23 &25 &16 &8 \\ 
		\hline
		User 7 &12 &20 &23 &15 &8  \\ 
		\hline
		User 8 &10 &21 &26 &15 &8  \\ 
		\hline
		Total(s)  &79 &164 &205 &127 &53 \\ 
		\hline
		Rata-Rata(s)  &10 &21 &26 &16 &7  \\ 
		\hline
	\end{tabular}
\end{table}

\begin{table}[H]
	\center
	\caption{\textit{Time On Task} Pengujian Aplikasi Berbasis Android untuk Pangkalan.}
	\label{timeOnTaskPangkalan}
	\begin{tabular}{|c|l|l|l|l|l|}
		\hline
		\multirow{2}{*}{Nama} & \multicolumn{5}{c|}{Tugas (s)} \\ \cline{2-6} 
		&1 &2  &3 &4 &5  \\
		\hline
		User 1 &12 &20 &24 &15 &7  \\ 
		\hline
		User 2 &11 &22 &26 &15 &8  \\ 
		\hline
		User 3 &10 &21 &27 &14 &8  \\ 
		\hline
		User 4 &9 &19 &24 &16 &10  \\ 
		\hline
		User 5 &12 &23 &25 &14 &9  \\ 
		\hline
		User 6 &10 &19 &28 &17 &10 \\ 
		\hline
		User 7 &9 &23 &27 &16 &7  \\ 
		\hline
		User 8 &10 &22 &21 &19 &9  \\ 
		\hline
		Total(s)  &83 &169 &202 &126 &68 \\ 
		\hline
		Rata-Rata(s)  &11 &22 &26 &16 &9  \\ 
		\hline
	\end{tabular}
\end{table}

Tugas 2 dan 3 pada tabel \ref{timeOnTaskAgen} dibutuhkan waktu yang sedikit lebih lama ketimbang tugas lainnya karena ada beberapa \textit{form} yang harus diisikan oleh \textit{user}. Begitu juga tugas 2 dan 3 pada tabel \ref{timeOnTaskPangkalan} dibutuhkan waktu agak lama untuk mengisi \textit{form} dan juga mengunggah data foto.

Penghitungan \textit{task error} dilakukan untuk mengetahui seberapa banyak kesalahan yang dilakukan pengguna dalam mengerjakan tugas yang diberikan.Hasil dari penghitungan nya  dapat dilihat pada tabel \ref{taskErrorAgen} untuk agen dan table \ref{taskErrorPangkalan} untuk pangkalan. 
\begin{table}[H]
	\center
	\caption{\textit{Task Error} Pengujian Aplikasi Berbasis web untuk Agen.}
	\label{taskErrorAgen}
	\begin{tabular}{|c|l|l|l|l|l|l|}
		\hline
		\multirow{2}{*}{Nama} & \multicolumn{5}{c|}{Tugas} &  \multirow{2}{0.5cm}{Jml} \\ \cline{2-6} 
		&1 &2  &3 &4 &5& \\
		\hline
		User 1 &0 &0 &0 &0 &0 &0 \\ 
		\hline
		User 2 &0 &0 &1 &0 &0 &1  \\ 
		\hline
		User 3 &0 &0 &0 &0 &0 &0  \\ 
		\hline
		User 4 &0 &0 &0 &0 &0 &0  \\ 
		\hline
		User 5 &0 &0 &0 &1 &0 &1  \\ 
		\hline
		User 6 &0 &0 &1 &1 &1 &2  \\ 
		\hline
		User 7 &0 &0 &0 &0 &0 &0 \\ 
		\hline
		User 8 &0 &0 &1 &0 &0 &1 \\ 
		\hline
		Total(s)  &0 &0 &3 &2 &0 &5 \\ 
		\hline
	\end{tabular}
\end{table}

\begin{table}[H]
	\center
	\caption{\textit{Task Error} Pengujian Aplikasi Berbasis android untuk Pangkalan.}
	\label{taskErrorPangkalan}
	\begin{tabular}{|c|l|l|l|l|l|l|}
		\hline
		\multirow{2}{*}{Nama} & \multicolumn{5}{c|}{Tugas} &  \multirow{2}{0.5cm}{Jml} \\ \cline{2-6} 
		&1 &2  &3 &4 &5& \\
		\hline
		User 1 &0 &0 &1 &0 &0 &1 \\ 
		\hline
		User 2 &0 &0 &1 &0 &0 &1  \\ 
		\hline
		User 3 &0 &0 &0 &0 &0 &0  \\ 
		\hline
		User 4 &0 &0 &0 &0 &0 &0  \\ 
		\hline
		User 5 &0 &1 &0 &0 &0 &1  \\ 
		\hline
		User 6 &0 &1 &1 &0 &0 &2  \\ 
		\hline
		User 7 &0 &0 &0 &0 &0 &0 \\ 
		\hline
		User 8 &0 &0 &0 &0 &1 &1 \\ 
		\hline
		Total(s)  &0 &2 &3 &0 &1 &6 \\ 
		\hline
	\end{tabular}
\end{table}
Tabel \ref{taskErrorAgen} dapat dilihat bahwa tugas 3 dan 4 sering terjadi \textit{task error}. Berdasarkan \textit{think aloud} yang dilakukan saat proses usability hal ini terjadi karena \textit{user} agen belum terbiasa dengan alur aplikasi. Sedangkan pada tabel \ref{taskErrorPangkalan} dapat dilihat bahwa tugas tugas 2 dan 3 sering terjadi \textit{task error}, ini  terjadi karena \textit{user} pangkalan belum terbiasa dengan tampilan aplikasi pada saat mengambil gambar.
Aplikasi ini dirancang dan dibangun dan diuji untuk pertama kalinya, sehingga pasti memiliki kritik  yang diberikan saat proses \textit{usability testing} dilakukan. Berikut beberapa kritik yang diberikan oleh pengguna : \newline \newline
Kritik :
\begin{enumerate}
	\item Fitur untuk impor data penerimaan pasokan sekaligus via CSV untuk pengembangan selanjutnya.
	\item Fitur untuk melakukan pemblokiran pada pelanggan yang telah melewati batas pembelian untuk pengembangan selanjutnya.
	\item Tampilan pada saat mengambil gambar di \textit{improve} lagi, tapi untuk fitur-fitur dasarnya sudah cukup lumayan.
	\item UX harus di \textit{improve} lagi untuk mempercepat proses validasi pada proses pembelian tabung dikarenakan adanya antrian pada saat membeli tabung.
	
\end{enumerate}  
	
% Baris ini digunakan untuk membantu dalam melakukan sitasi.
% Karena diapit dengan comment, maka baris ini akan diabaikan
% oleh compiler LaTeX.
\begin{comment}
\bibliography{daftar-pustaka}
\end{comment}

%-------------------------------------------------------------------------------
%                            	BAB V
%               		KESIMPULAN DAN SARAN
%-------------------------------------------------------------------------------

\chapter{KESIMPULAN DAN SARAN}

\section{Kesimpulan}
	Kesimpulan pada penelitian ini adalah sebagai berikut:

	\begin{enumerate}
		\item Pengembangan aplikasi ini dirancang berdasarkan masalah atau \textit{problem-based analysis} dengan menggunakan \textit{problem frames}. 
		\item Untuk mencapai \textit{problem frames}, tahap yang dilakukan terdiri dari membuat \textit{user stories}, \textit{requirement}, menentukan domain dan \textit{share phenomena}.
		\item Aplikasi yang dikembangkan terbagi menjadi 2 bagian yaitu aplikasi web untuk pemilik kos dan admin dan aplikasi Android untuk pencari kos. 
		\item Pemilik kos dapat mempromosikan kos miliknya melalui aplikasi web jika diverifikasi oleh admin.
		\item Pencari kos dapat mencari dan memesan kos dari aplikasi Android tanpa harus berkeliling Banda Aceh untuk mencari informasi kos.
		\item Aplikasi Android yang dikembangkan menggunakan fitur QR Code untuk melihat informasi kos tanpa harus bertanya langsung pada pemilik kos.
		\item Berdasarkan pengujian \textit{usability} dengan metode \textit{System Usability Scale} (SUS), aplikasi web untuk pemilik kos mendapatkan skor SUS 72,5 sedangkan aplikasi Android untuk pencari kos mendapatkan skor 80,6, dimana kedua nilai tersebut masuk pada kategori \textit{Acceptable} yaitu dapat diterima oleh pengguna akhir.
		
	\end{enumerate}


\section{Saran}

	Penelitian ini masih banyak kekurangan sehingga perlu dikembangkan agar menjadi lebih baik. Berikut adalah saran untu penelitian ini:
	\begin{enumerate}
		\item Aplikasi yang digunakan oleh pemilik kos dan pencari kos tersedia dalam web, Android dan iOS.
		\item Tampilan dari aplikasi dapat diperbaiki lagi menjadi lebih menarik dan \textit{user friendly}. Agar dapat menaikkan \textit{SUS Score} terutama pada aplikasi web.
		\item Pada aplikasi Android, dapat ditambah fitur \textit{bookmark}. Sehingga tidak perlu mencari kos yang dimaksud di halaman Lihat Semua Kos. 
	\end{enumerate}

	
% Baris ini digunakan untuk membantu dalam melakukan sitasi
% Karena diapit dengan comment, maka baris ini akan diabaikan
% oleh compiler LaTeX.
\begin{comment}
\bibliography{daftar-pustaka}
\end{comment}


%%-------------------------------------------------------------------------------
%                            	BAB V
%               		KESIMPULAN DAN SARAN
%-------------------------------------------------------------------------------

\chapter{KESIMPULAN DAN SARAN}

\section{Kesimpulan}
	Kesimpulan pada penelitian ini adalah sebagai berikut:

	\begin{enumerate}
		\item Pengembangan aplikasi ini dirancang berdasarkan masalah atau \textit{problem-based analysis} dengan menggunakan \textit{problem frames}. 
		\item Untuk mencapai \textit{problem frames}, tahap yang dilakukan terdiri dari membuat \textit{user stories}, \textit{requirement}, menentukan domain dan \textit{share phenomena}.
		\item Aplikasi yang dikembangkan terbagi menjadi 2 bagian yaitu aplikasi web untuk pemilik kos dan admin dan aplikasi Android untuk pencari kos. 
		\item Pemilik kos dapat mempromosikan kos miliknya melalui aplikasi web jika diverifikasi oleh admin.
		\item Pencari kos dapat mencari dan memesan kos dari aplikasi Android tanpa harus berkeliling Banda Aceh untuk mencari informasi kos.
		\item Aplikasi Android yang dikembangkan menggunakan fitur QR Code untuk melihat informasi kos tanpa harus bertanya langsung pada pemilik kos.
		\item Berdasarkan pengujian \textit{usability} dengan metode \textit{System Usability Scale} (SUS), aplikasi web untuk pemilik kos mendapatkan skor SUS 72,5 sedangkan aplikasi Android untuk pencari kos mendapatkan skor 80,6, dimana kedua nilai tersebut masuk pada kategori \textit{Acceptable} yaitu dapat diterima oleh pengguna akhir.
		
	\end{enumerate}


\section{Saran}

	Penelitian ini masih banyak kekurangan sehingga perlu dikembangkan agar menjadi lebih baik. Berikut adalah saran untu penelitian ini:
	\begin{enumerate}
		\item Aplikasi yang digunakan oleh pemilik kos dan pencari kos tersedia dalam web, Android dan iOS.
		\item Tampilan dari aplikasi dapat diperbaiki lagi menjadi lebih menarik dan \textit{user friendly}. Agar dapat menaikkan \textit{SUS Score} terutama pada aplikasi web.
		\item Pada aplikasi Android, dapat ditambah fitur \textit{bookmark}. Sehingga tidak perlu mencari kos yang dimaksud di halaman Lihat Semua Kos. 
	\end{enumerate}

	
% Baris ini digunakan untuk membantu dalam melakukan sitasi
% Karena diapit dengan comment, maka baris ini akan diabaikan
% oleh compiler LaTeX.
\begin{comment}
\bibliography{daftar-pustaka}
\end{comment}


%-----------------------------------------------------------------
% Disini akhir masukan Bab
%-----------------------------------------------------------------

%-----------------------------------------------------------------
% Disini awal masukan untuk Daftar Pustaka
% - Daftar pustaka diambil dari file .bib yang ada pada folder ini
%   juga.
% - Untuk memudahkan dalam memanajemen dan menggenerate file .bib
%   gunakan reference manager seperti Mendeley, Zotero, EndNote,
%   dll.
%-----------------------------------------------------------------
\begin{spacing}{1}
\bibliography{daftar-pustaka}
\end{spacing}
\addcontentsline{toc}{chapter}{DAFTAR KEPUSTAKAAN}
%-----------------------------------------------------------------
% Disini akhir masukan Daftar Pustaka
%-----------------------------------------------------------------

%-----------------------------------------------------------------------------%
\addcontentsline{toc}{chapter}{LAMPIRAN}
\chapter*{Lampiran}
\begin{landscape}
	\begin{flushleft}
		\newappendix{Lampiran 1. Hasil Kuesioner SUS Aplikasi Scan Kos web yang sudah dikonversi}
	\end{flushleft}
	
	\begin{longtable}{|c|c|c|c|c|c|c|c|c|c|c|c|}
		
		\hline
		\multirow{2}{*}{Responden} & \multicolumn{10}{c|}{Pernyataan SUS} & 
		\multirow{2}{*}{Jumlah} \\
		
		\cline{2-11} & No.1 & No.2 & No.3 & No.4 & No.5 & No.6 & No.7 & No.8 & No.9 & No.10 &\\ \hline 
		1 & 2 & 2 & 3 & 2 & 3 & 3 & 3 & 3 & 3 & 1 & 25 \\ \hline
		2 & 3 & 3 & 4 & 4 & 3 & 3 & 2 & 3 & 3 & 3 &31 \\ \hline
		3 & 3 &3& 4& 1& 2& 3& 2& 3& 2& 0& 23 \\ \hline
		4 & 3 &3& 4& 2& 2& 2& 3& 2& 3& 3& 28 \\ \hline
		5 & 4 &3& 3& 3& 2& 4& 4& 4& 2& 2& 31 \\ \hline
		6 & 4 &3& 3& 4& 3& 4& 2& 3& 4& 2& 32 \\ \hline
		7 & 3 &1& 2& 4& 1& 3& 1& 3& 3& 3& 24 \\ \hline
		8 & 4 &3& 3& 3& 4& 4& 4& 2& 3& 3& 33 \\ \hline
		9 & 3 &4& 3& 4& 4& 3& 2& 4& 2& 3& 32 \\ \hline
		10& 3 &3& 4& 4& 3& 3& 2& 2& 4& 3& 31 \\ \hline
		11& 2 &2& 3& 2& 3& 4& 3& 3& 3& 1& 26 \\ \hline
		12& 4 &3& 4& 4& 3& 3& 2& 3& 3& 1& 30 \\ \hline
		13& 2 &3& 4& 1& 2& 3& 2& 3& 2& 1& 23 \\ \hline
		14& 3 &4& 4& 2& 2& 2& 3& 2& 4& 3& 29 \\ \hline
		15& 3 &4& 3& 4& 4& 3& 2& 4& 2& 3& 32 \\ \hline
		16& 4 &3& 4& 1& 3& 4& 2& 3& 4& 1& 29 \\ \hline
		17& 4 &3& 3& 3& 4& 4& 4& 2& 3& 3& 33 \\ \hline
		18& 3 &1& 2& 4& 1& 3& 1& 3& 3& 3& 24 \\ \hline
		19& 3 &3& 4& 4& 3& 3& 2& 2& 4& 3& 31 \\ \hline
		20& 4 &3& 3& 3& 2& 4& 4& 4& 2& 1& 30 \\ \hline
		21& 3 &4& 3& 4& 4& 3& 2& 4& 2& 3& 32 \\ \hline
		22& 4 &3& 4& 1& 3& 4& 2& 3& 4& 1& 29 \\ \hline
		23& 4 &3& 3& 4& 3& 4& 2& 3& 4& 2& 32 \\ \hline
		24& 3 &3& 4& 4& 3& 3& 2& 3& 3& 3& 31 \\ \hline
		25& 3 &3& 4& 1& 2& 3& 2& 3& 2& 1& 24 \\ \hline
		26& 4 &3& 4& 4& 3& 3& 2& 3& 3& 1& 30 \\ \hline
		27& 3 &1& 2& 4& 1& 3& 1& 3& 3& 3& 24 \\ \hline
		28& 2 &3& 4& 1& 2& 3& 2& 3& 2& 1& 23 \\ \hline
		29& 3 &4& 4& 2& 2& 2& 3& 2& 4& 3& 29 \\ \hline
		30& 4 &3& 4& 4& 3& 3& 4& 4& 3& 4& 36 \\ \hline
		
		
		
		
		\hline \multicolumn{11}{|c|}{Rata-rata Skor}  & 33,74\\
		\hline \multicolumn{11}{|c|}{Skor SUS = 33.74 * 2.5}  & 84,36\\
		\hline
	\end{longtable}
\end{landscape}

\begin{landscape}
	\begin{flushleft}
		\newappendix{Lampiran 2. Hasil Kuesioner SUS Aplikasi Scan Kos Android yang sudah dikonversi}
	\end{flushleft}
	
	\begin{longtable}{|c|c|c|c|c|c|c|c|c|c|c|c|}
		
		\hline
		\multirow{2}{*}{Responden} & \multicolumn{10}{c|}{Pernyataan SUS} & 
		\multirow{2}{*}{Jumlah} \\
		
		\cline{2-11} & No.1 & No.2 & No.3 & No.4 & No.5 & No.6 & No.7 & No.8 & No.9 & No.10 &\\ \hline 
		1 & 3& 4& 4& 4& 3& 3& 4& 4& 3& 3& 35 \\ \hline
		2 & 2& 1& 2& 1& 2& 3& 3& 2& 2& 1& 19 \\ \hline
		3 & 4& 3& 3& 3& 4& 4& 4& 3& 3& 2& 33  \\ \hline
		4 & 4& 2& 4& 3& 4& 4& 4& 4& 4& 3& 36 \\ \hline
		5 & 4& 3& 4& 4& 4& 4& 3& 4& 4& 3& 32  \\ \hline
		6 & 4& 4& 4& 3& 4& 4& 4& 4& 4& 3& 38 \\ \hline
		7 & 3& 3& 4& 2& 3& 4& 3& 3& 3& 3& 31 \\ \hline
		8 & 2& 4& 4& 3& 3& 3& 3& 4& 4& 4& 34 \\ \hline
		9 & 4& 1& 2& 4& 1& 3& 1& 3& 3& 3& 25 \\ \hline
		10& 3& 2& 4& 3& 3& 3& 2& 3& 3& 4& 30 \\ \hline
		11& 4& 3& 4& 4& 3& 3& 4& 4& 3& 4& 36 \\ \hline
		12& 4& 4& 2& 1& 2& 3& 3& 2& 2& 1& 24 \\ \hline
		13& 4& 3& 3& 3& 2& 4& 4& 4& 2& 2& 31 \\ \hline
		14& 4& 2& 4& 3& 4& 2& 4& 1& 3& 3& 30  \\ \hline
		15& 4& 3& 3& 4& 3& 4& 3& 3& 4& 3& 34 \\ \hline
		16& 4& 4& 2& 1& 2& 3& 3& 2& 2& 1& 24 \\ \hline
		17& 3& 3& 3& 3& 2& 4& 3& 3& 3& 2& 29 \\ \hline
		18& 4& 4& 3& 4& 2& 4& 2& 3& 4& 3& 33 \\ \hline
		19& 4& 3& 3& 3& 2& 3& 3& 4& 4& 3& 32 \\ \hline
		20& 4& 4& 4& 3& 3& 3& 3& 4& 4& 4& 36 \\ \hline
		21& 4& 3& 4& 3& 4& 3& 3& 4& 2& 4& 34 \\ \hline
		22& 2& 4& 4& 4& 3& 3& 4& 4& 3& 3& 34 \\ \hline
		23& 4& 4& 4& 3& 4& 4& 4& 4& 4& 3& 38 \\ \hline
		24& 4& 4& 3& 3& 4& 3& 3& 3& 3& 2& 32\\ \hline
		25& 4& 4& 4& 3& 4& 4& 4& 4& 4& 3& 38 \\ \hline
		26& 4& 3& 4& 4& 3& 3& 4& 4& 3& 4& 36 \\ \hline
		27& 4& 2& 4& 3& 4& 4& 4& 4& 4& 3& 36 \\ \hline
		28& 4& 2& 4& 3& 4& 2& 4& 1& 3& 3& 30 \\ \hline
		29& 4& 3& 3& 4& 3& 4& 3& 3& 4& 3& 34 \\ \hline
		30& 4& 3& 4& 4& 4& 4& 3& 4& 4& 3& 32  \\ \hline
		
		
		
		
		\hline \multicolumn{11}{|c|}{Rata-rata Skor}  & 33,74\\
		\hline \multicolumn{11}{|c|}{Skor SUS = 33.74 * 2.5}  & 84,36\\
		\hline
	\end{longtable}
\end{landscape}
\addcontentsline{toc}{chapter}{LAMPIRAN} %daftar lampiran

%Lampiran 4 Laporan Usability
\includepdf[pages=-]{laporan_usability}

\end{onehalfspace}

\end{document}