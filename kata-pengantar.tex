\preface % Note: \preface JANGAN DIHAPUS!


\par Syukur Alhamdulillah penulis panjatkan ke hadirat Allah SWT yang telah melimpahkan rahmat-Nya sehingga penulis dapat menyelesaikan Tugas Akhir yang berjudul \textbf{“Rancang Bangun Aplikasi Pelaporan Penyaluran Gas LPG 3 Kg Pada Pangkalan Gas Lpg”}. Shalawat dan salam tak lupa pula penulis sanjungkan kepada junjungan kita Nabi Muhammad SAW yang telah menerangi alam ini dengan cahaya keislaman dan ilmu pengetahuan.
\par Tugas Akhir ini merupakan salah satu syarat yang harus dipenuhi untuk memperoleh gelar Sarjana Komputer di Jurusan Informatika, Fakultas Matematika dan Ilmu Pengetahuan Alam, Universitas Syiah Kuala. Penulis menyadari bahwa penulisan Tugas Akhir ini tidak terlepas dari bantuan dan dorongan dari berbagai pihak, baik secara moril maupun materil. Pada kesempatan ini penulis ingin mengucapkan terima kasih kepada:

\begin{enumerate}
	\item{Bapak Dr. Teuku Mohamad Iqbalsyah, S.Si, M.Sc, selaku Dekan Fakultas MIPA}
	\item{Bapak Dr. Muhammad Subianto, S.Si, M.Si, selaku Ketua Jurusan Informatika Fakultas MIPA Universitas Syiah Kuala.}
	\item{Bapak Irvanizam, S.Si, M.Sc, selaku Dosen Wali.}
	\item{Bapak Rahmad Dawood, S.Kom, M.Sc., selaku dosen pembimbing I dan Bapak Kurnia Saputra, S.T, M.Sc, selaku dosen pembimbing II yang telah banyak memberikan bimbingan dan arahan kepada penulis dalam penyelesaian Proposal Penelitian Tugas Akhir ini.,}
	\item{Dengan tidak mengurangi rasa hormat penulis ucapkan terima kasih sebesar- besarnya kepada Ayahanda Saya Alm. Darnius dan Ibunda saya Ida Nursanti yang tak henti-henti mendukung, memberikan motivasi dan senantiasa mendoakan penulis dari awal masa studi hingga saat ini.}
	\item{Seluruh Dosen di Jurusan Informatika Fakultas MIPA, yang tidak bisa disebutkan satu-satu, atas ilmu dan didikannya selama perkuliahan,}
	\item {Staf administrasi Jurusan Informatika yang telah membantu proses administrasi.}
	\item {Sahabat seperjuangan Riki dan Yusran } 
	\item{Seluruh teman-teman jurusan Informatika Unsyiah angkatan 2014 sampai 2016 yang telah memberikan semangat kepada penulis dalam menyelesaikan Tugas Akhir ini.}
\end{enumerate}

\newpage
Dengan segala kerendahan hati, penulis mengharapkan semoga tulisan ini
dapat bermanfaat bagi para pembaca, peneliti dan bagi perkembangan ilmu
pengetahuan.

\vspace{0.5cm}


\begin{tabular}{p{7.5cm}l}
	&Banda Aceh, Juli 2019\\
	&\\
	&\\
	&\underline{Andri Darnius}\\
	&NPM.1608107010057
\end{tabular}